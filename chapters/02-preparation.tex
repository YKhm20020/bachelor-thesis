\chapter{研究の準備}\label{cha:Preparation}
(簡単なメモ。入力画像についてのみ明記。あとで詳しく書くこと)

(3章の画像処理についての説明が長くなっている。これ以上見出しを増やすと番号が4.1.1.1のように冗長になるので、説明を一部移す予定)

本章では、本研究に必要な前提知識について説明する。


\section{入力対象の画像}\label{sec:input_images}
本提案手法の入力対象となる画像については、以下の条件を全て満たす画像とする。
なお、帳票は帳簿と伝票の総称であり\cite{帳票}、電子文書は、Wordやテキストファイルなど、デジタル情報として作成した文書を指し、電子化文書は紙媒体の書類をスキャンしたPDFファイルや画像などとして保存した文書を指す\cite{電子文書と電子化文書}。
電子化文書画像については、本提案手法においては、スマートフォンのカメラで撮影した帳票の画像を想定している。

\begin{itemize}
  \item 帳票の電子文書または電子化文書の画像である。
  \item 日本語かつ横書きの帳票である。
  \item 文字が手書きではない。
  \item 入力欄が矩形または下線で示されている。
\end{itemize}

なお、入力対象とする画像の座標系は、画像左上隅画素を原点(0,0)とし、右方向をX軸の正の向き、下方向をY軸の正の向きと定義する。

\section{バリデーションチェック}\label{sec:validation_check}


\section{OpenCV}\label{sec:OpenCV}



\section{DeblurGANv2}\label{sec:Deblur-GANv2}



\section{光学文字認識}\label{sec:Optical-Charactor-Recognition}


\section{Fugashi}\label{sec:Fugashi}



\section{Youri}\label{sec:Youri}
