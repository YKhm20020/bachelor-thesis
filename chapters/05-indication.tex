\chapter{適用例}\label{cha:Indication}
本章では、本研究で提案した手法が正しく動作することを確認する。
適用例として、本提案手法を適用する帳票画像を、図\ref{fig:indication_original}に示す。

\begin{figure}[t]
    \begin{center}
        \fbox{
            \includegraphics[width=15cm]{image/05-indication/indication_original.jpg}
        }
        \caption{本提案手法を適用する帳票画像}
        \label{fig:indication_original}
    \end{center}
\end{figure}

図\ref{fig:indication_original}に対して、本提案手法を適用し、出力であるJSONファイルと、2枚の領域強調画像を確認する。
具体的には、矩形領域強調画像と、JSONファイルのrects\_data配列を参照し、矩形領域の出力結果を確認する。
同様に、下線部領域強調画像と、JSONファイル内のunderlines\_data配列を参照し、下線部領域の出力結果を確認する。

図\ref{fig:indication_original}に対して、本提案手法を適用し、出力したJSONファイルのうち、rects\_data配列の一部を、図\ref{fig:rects_data_json}に示す。
また、図\ref{fig:rects_data_json}のJSONファイルと同時に出力した2枚の領域強調画像のうち、矩形領域を強調した画像の一部を図\ref{fig:highlighted_rects_part}に示す。

\lstset{language=}
\begin{figure}[t]
    \begin{lstlisting}
        {
            "id": 4,
            "label": "string",
            "coords": {
                "top_left": {
                    "x": 275,
                    "y": 817
                },
                "buttom_left": {
                    "x": 275,
                    "y": 903
                },
                "buttom_right": {
                    "x": 1008,
                    "y": 903
                },
                "top_right": {
                    "x": 1008,
                    "y": 817
                }
            }
        },
        {
            "id": 5,
            "label": "number",
            "coords": {
                "top_left": {
                    "x": 1016,
                    "y": 817
                },
                "buttom_left": {
                    "x": 1016,
                    "y": 903
                },
                "buttom_right": {
                    "x": 1308,
                    "y": 903
                },
                "top_right": {
                    "x": 1308,
                    "y": 817
                }
            }
        },
    \end{lstlisting}
    \caption{rects\_data配列の一部}\label{fig:rects_data_json}
\end{figure}

\begin{figure}[t]
    \begin{center}
        \fbox{
            \includegraphics[width=15cm]{image/05-indication/highlighted_rects_part.jpg}
        }
        \caption{矩形領域を強調した画像の一部}
        \label{fig:highlighted_rects_part}
    \end{center}
\end{figure}

図\ref{fig:indication_original}に対して、本提案手法を適用し、出力したJSONファイルのうち、underline\_data配列の一部を、図\ref{fig:underlines_data_json}に示す。
また、図\ref{fig:underlines_data_json}のJSONファイルと同時に出力した2枚の下線部強調画像のうち、下線部領域を強調した画像の一部を図\ref{fig:highlighted_underlines_part}に示す。

\lstset{language=}
\begin{figure}[t]
    \begin{lstlisting}
        {
            "id": 0,
            "label": "date",
            "left": {
                "x": 869,
                "y": 354
            },
            "right": {
                "x": 1512,
                "y": 354
            }
        },
        {
            "id": 1,
            "label": "number",
            "left": {
                "x": 1908,
                "y": 355
            },
            "right": {
                "x": 2265,
                "y": 355
            }
        },
    \end{lstlisting}
    \caption{underline\_data配列の一部}\label{fig:underlines_data_json}
\end{figure}

\begin{figure}[t]
    \begin{center}
        \fbox{
            \includegraphics[width=15cm]{image/05-indication/highlighted_underlines_part.jpg}
        }
        \caption{下線部領域を強調した画像の一部}
        \label{fig:highlighted_underlines_part}
    \end{center}
\end{figure}

\section{矩形領域についての出力結果}\label{sec:result_rect}
本節は、矩形領域についての出力結果を確認する。
図\ref{fig:rects_data_json}より、idキーに対応する値は4、5となっており、それぞれラベルはstring、numberとなっている。
図\ref{fig:highlighted_rects_part}より、画像内に描画した矩形領域の番号のうち、4番と5番の矩形領域は、品名、および、数量を記入する欄であることがわかる。
品名、数量という記入欄については、それぞれ文字列(string)、数値(number)が正しいラベルであるため、これら2つの矩形領域については、正しくラベルを割り付けていることを確認できた。
他の矩形領域に対しても、一部を除き、正しくラベルを割り付けていることを確認した。

一部の矩形領域については、本来割り付けるべきラベルとは異なるラベルを誤って割り付けた。
JSONファイルのうち、誤ったラベルを割り付けた矩形領域座標を、図\ref{fig:rects_data_miss_json}に示す。
また、図\ref{fig:rects_data_miss_json}に示した箇所の下線部領域強調画像を、図\ref{fig:highlighted_rects_miss_part}に示す。

\lstset{language=}
\begin{figure}[t]
    \begin{lstlisting}
        {
            "id": 25,
            "label": "string",
            "coords": {
                "top_left": {
                    "x": 1713,
                    "y": 1283
                },
                "buttom_left": {
                    "x": 1713,
                    "y": 1369
                },
                "buttom_right": {
                    "x": 2262,
                    "y": 1369
                },
                "top_right": {
                    "x": 2262,
                    "y": 1283
                }
            }
        },
    \end{lstlisting}
    \caption{誤ったラベルを割り付けた矩形領域座標}\label{fig:rects_data_miss_json}
\end{figure}

\begin{figure}[t]
    \begin{center}
        \fbox{
            \includegraphics[width=15cm]{image/05-indication/highlighted_rects_miss_part.jpg}
        }
        \caption{図\ref{fig:rects_data_miss_json}の矩形領域を描画した矩形領域強調画像}
        \label{fig:highlighted_rects_miss_part}
    \end{center}
\end{figure}

図\ref{fig:rects_data_miss_json}は、idキーに対応する値が25の矩形領域は、labelキーに対応する値がstringであることを示す。
図\ref{fig:highlighted_rects_miss_part}の画像を確認すると、25番の矩形領域は、小計を記入する欄であることがわかる。
本来割り付けるべきラベルは数値(number)であるため、誤ったラベルを割り付けていることがわかる。
この問題点については、\ref{sec:problems}節で後述する。


\section{下線部領域についての出力結果}\label{sec:result_underline}
図\ref{fig:underlines_data_json}より、idキーに対応する値は0、1となっており、それぞれラベルはdate、numberとなっている。
図\ref{fig:highlighted_underlines_part}より、0番と1番の下線部領域は、年月日、および、No.を記入する欄であることがわかる。
年月日、No.という記入欄については、それぞれ日付(date)、数値(number)が正しいラベルであるため、これら2つの下線部領域については、正しくラベルを割り付けていることを確認できた。
他の下線部領域に対しても、一部を除き、正しくラベルを割り付けていることを確認した。

一部の下線部領域については、本来割り付けるべきラベルとは異なるラベルを誤って割り付けた。
JSONファイルのうち、誤ったラベルを割り付けた下線部領域座標を、図\ref{fig:underlines_data_miss_json}に示す。
図\ref{fig:underlines_data_miss_json}の下線部領域は、図\ref{fig:highlighted_underlines_part}の2番の下線部領域である。

\lstset{language=}
\begin{figure}[t]
    \begin{lstlisting}
        {
            "id": 2,
            "label": "date",
            "left": {
                "x": 273,
                "y": 615
            },
            "right": {
                "x": 1312,
                "y": 615
            }
        },
    \end{lstlisting}
    \caption{誤ってラベルを割り付けた下線部領域}\label{fig:underlines_data_miss_json}
\end{figure}

図\ref{fig:underlines_data_miss_json}は、idキーに対応する値が2の下線部領域は、labelキーに対応する値がdateであることを示す。
図\ref{fig:highlighted_rects_miss_part}の画像を確認すると、2番の下線部領域は、宛名を記入する欄であることがわかる。
本来割り付けるべきラベルは文字列(string)であるため、誤ったラベルを割り付けていることがわかる。
この問題点については、\ref{sec:problems}節で後述する。