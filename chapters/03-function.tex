\chapter{試作するツールの機能}\label{cha:Function}
本章では、試作するツールの機能について説明する。
試作するツールは、\ref{sec:input_images}節で述べた帳票画像を入力とする。
出力は、領域の座標とそれに対応するラベルを組とするJSON(JavaScript Object Notation)形式のファイルと、領域の座標を帳票画像に描画したPNG(Portable Network Graphics)形式の画像である。
出力する画像は、矩形の領域を描画した画像と、下線部の領域を描画した画像の2枚に分けている。
これは、領域を誤検出した際に、2つの領域が重なることで視認性が低下することを防ぐためである。
なお、本研究で用いる用語を、表\ref{tb:term}に示す。
\begin{table}[tp]
	\caption{本研究で用いる用語}
	\label{tb:term}
	\centering
	\begin{tabular}{|c||l|}
        \hline
		用語 & \multicolumn{1}{c|}{意味} \\
        \hline \hline
		帳票画像内記入欄 & 入力である帳票画像内の記入欄 \\
        \hline
		電子フォーム内記入欄 & 試作するツールによって取得した座標で示す記入欄 \\
        \hline
		矩形領域 & 電子フォーム内記入欄のうち、矩形で示された記入欄 \\
        \hline
		下線部領域 & 電子フォーム内記入欄のうち、下線で示された記入欄 \\
        \hline
        \multirow{3}{*}{領域座標} & 試作するツールによって取得する領域の座標。 \\
                                    & 矩形領域であれば、各頂点4つのxy座標にあたる。 \\
                                    & 下線部領域であれば、両端点の2つのxy座標にあたる。 \\
        \hline
        \multirow{2}{*}{文字情報} & 光学文字認識によって認識した文字と \\
                                    & それを囲むバウンディングボックスの各頂点のxy座標 \\
        \hline
        \multirow{2}{*}{属性} & 光学文字認識によって得た文字から推測するデータ型。 \\
                                & 本研究では、日付(date)、文字列(string)、数値(number)の3つとする。\\
        \hline
        \multirow{2}{*}{ラベル} & 文字の近傍に存在する電子フォーム内記入欄に対して \\
                                & 文字位置と属性から推測するデータ型 \\
        \hline
	\end{tabular}
\end{table}

本研究で試作するツールは、以下の2つの機能を持つ。

\begin{itemize}
    \item 領域座標自動取得およびラベル割付機能
    \item 領域強調画像出力機能
\end{itemize}

試作するツールの入力対象である帳票画像の例を、図\ref{fig:original}に示す。
\begin{figure}[tp]
    \begin{center}
        \fbox{
            \includegraphics[width=15cm]{image/03-function/original.jpg}
        }
        \caption{入力対象である帳票画像の例}
        \label{fig:original}
    \end{center}
\end{figure}

また、図\ref{fig:original}に対して、試作するツールを適用し、出力したJSONファイルの一部を、図\ref{fig:example_output_json}に示す。
\lstset{language=}
\begin{figure}[tp]
    \begin{lstlisting}
    {
        "rects_data": [
            {
                "id": 0, 
                "label": "number",
                "coords": {
                    "top_left": {
                        "x": 1339,
                        "y": 1288
                        },
                    "bottom_left": {
                        "x": 1339,
                        "y": 1377
                        },
                    "bottom_right": {
                        "x": 1782,
                        "y": 1377
                        },
                    "top_right": {
                        "x": 1782,
                        "y": 1288
                    }
                }
            }
        ],
        "underlines_data": [
            {
                "id": 0,
                "label": "string",
                "coords": {
                    "left": {
                        "x": 213,
                        "y": 742
                        },
                    "right": {
                        "x": 1111,
                        "y": 742
                    }
                }
            }
        ]
    }
    \end{lstlisting}
    \caption{出力するJSONファイルの一部}\label{fig:example_output_json}
\end{figure}

なお、出力するJSONファイルは、矩形領域のラベルと領域座標をまとめたrects\_data配列と、下線部領域のラベルと領域座標をまとめたunderlines\_data配列で構成する。
rects\_data配列は、以下のキーとオブジェクトから構成する。

\begin{itemize}
    \item 取得した矩形の領域座標のうち、どの領域座標かを一意に定める番号を示すidキー
    \item 領域座標に割り付けたラベルを示すlabelキー
    \item 矩形の領域座標をまとめたcoordsオブジェクト
    \begin{itemize}
        \item 左上頂点のxy座標を示し、xキーとyキーにそれぞれx座標とy座標が対応するtop\_leftオブジェクト
        \item 左下頂点のxy座標を示し、xキーとyキーにそれぞれx座標とy座標が対応するbottom\_leftオブジェクト
        \item 右下頂点のxy座標を示し、xキーとyキーにそれぞれx座標とy座標が対応するbottom\_rightオブジェクト
        \item 右上頂点のxy座標を示し、xキーとyキーにそれぞれx座標とy座標が対応するtop\_rightオブジェクト
    \end{itemize}
\end{itemize}

また、underlines\_data配列は、以下のキーとオブジェクトから構成する。

\begin{itemize}
    \item 取得した下線部の領域座標のうち、どの領域座標かを一意に定める番号を示すidキー
    \item 領域座標に割り付けたラベルを示すlabelキー
    \item 下線部の領域座標をまとめたcoordsオブジェクト
    \begin{itemize}
        \item 左端点のxy座標を示し、xキーとyキーにそれぞれx座標とy座標が対応するleftオブジェクト
        \item 右端点のxy座標を示し、xキーとyキーにそれぞれx座標とy座標が対応するrightオブジェクト
    \end{itemize}
\end{itemize}

\section{領域座標自動取得およびラベル割付機能}\label{sec:eform_write_space_obtainment_feature}
領域座標自動取得およびラベル割付機能は、帳票画像中の矩形領域および下線部領域について、それらの領域座標を電子フォーム内記入欄として取得し、それぞれにラベルを割り付ける機能である。
ラベルを割り付けることで、バリデーションチェック(\ref{sec:validation_check}節を参照)に必要な情報を付与できる。

電子フォーム内記入欄の取得とラベルの割り付けは、以下の手順で行う。

\begin{enumerate}
    \item 矩形領域取得
    \item 下線部領域取得
    \item 文字情報取得
    \item 属性推測
    \item ラベル割付
\end{enumerate}

\subsection{矩形領域取得}\label{subsec:rect_coords_obtainment}
矩形領域取得では、矩形を帳票画像内記入欄とみなし、各頂点の4つのxy座標を領域座標として取得する。
以降、下線部の領域座標と区別するため、矩形領域の領域座標を、矩形領域座標と呼ぶ。
取得した全ての矩形領域座標を、左上頂点のy座標について昇順にソートし、0から順に番号を割り振る。
各矩形領域について、割り振った番号と領域座標は、出力するJSONファイルにおいて、それぞれrects\_data配列内のidキーとcoordsオブジェクトとする。

\subsection{下線部領域取得}\label{subsec:underline_coords_obtainment}
下線部領域取得では、水平な直線を帳票画像内記入欄とみなし、両端点の2つのxy座標を領域座標として取得する。
以降、矩形の領域座標と区別するため、下線部領域の領域座標を、下線部領域座標と呼ぶ。
取得した全ての下線部領域座標を、左端点のy座標について昇順にソートし、0から順に番号を割り振る。
各下線部領域について、割り振った番号と領域座標は、出力するJSONファイルにおいて、それぞれunderlines\_data配列内のidキーとcoordsオブジェクトとする。

\subsection{文字情報取得}\label{subsec:char_information_obtainment}
文字情報取得では、入力である帳票画像に対して光学文字認識を行い、検出した文字を取得文字とし、検出した文字を囲むバウンディングボックスの各頂点の座標を文字位置として、それぞれ取得する。
取得文字と文字位置は、それぞれ属性推測(\ref{subsec:att_prediction}節で後述)と、ラベル割付(\ref{subsec:label_link}節で後述)で用いる。

\subsection{属性推測}\label{subsec:att_prediction}
属性推測では、取得文字に対して、大規模言語モデルYouriによる属性の推測を行う。
本研究で用いる3つの属性を、以下に示す。

\begin{itemize}
    \item 日付(date)
    \item 文字列(string)
    \item 数値(number)
\end{itemize}

取得文字の属性が、上記3つのうちのいずれに該当するかを推測し、属性として取得する。
取得した属性は、ラベル割付(\ref{subsec:label_link}節で後述)で用いる。

\subsection{ラベル割付}\label{subsec:label_link}
矩形領域取得(\ref{subsec:rect_coords_obtainment}節を参照)と下線部領域取得(\ref{subsec:underline_coords_obtainment}節を参照)で取得した領域座標、文字情報取得(\ref{subsec:char_information_obtainment}節を参照)で取得した文字位置、属性推測(\ref{subsec:att_prediction}節を参照)で取得した属性を用いて、文字位置の近傍に存在する領域座標を対象に属性を割り付け、ラベルとして取得する。
出力するJSONファイルにおける、rects\_data配列、およびunderlines\_data配列内の、labelキーの値を取得する。

\section{領域強調画像出力機能}\label{sec:highlighted_area_image_output}
領域強調画像出力機能は、入力である帳票画像に対して、取得した領域とラベルを描画することによって、取得した領域を強調表示したPNG画像を出力する機能である。
これによって、\ref{sec:eform_write_space_obtainment_feature}節で述べた、領域座標自動取得およびラベル割付機能で出力したJSONファイルの内容を、目視で確認しやすくする。
本機能で出力する画像は、矩形領域を強調した矩形領域強調画像と、下線部領域を強調した下線部領域強調画像の計2枚である。
矩形領域強調画像は、ランダムな色で矩形を描画し、下線部領域強調画像は、緑色で直線を描画する。
矩形をランダムな色で描画する理由は、矩形領域が隣接する際に、視認性が低下することを防ぐためである。

図\ref{fig:original}に示した帳票画像内記入欄のうち、矩形の帳票画像内記入欄の一部を切り取った画像を、図\ref{fig:rect_original}に示す。
\begin{figure}[tp]
    \begin{center}
        \fbox{
            \includegraphics[width=15cm]{image/03-function/rect_original.jpg}
        }
        \caption{帳票画像にある矩形の記入欄}
        \label{fig:rect_original}
    \end{center}
\end{figure}
また、図\ref{fig:rect_original}の画像に対し、本機能を適用した画像を、図\ref{fig:rect_drawing}に示す。
\begin{figure}[tp]
    \begin{center}
        \fbox{
            \includegraphics[width=15cm]{image/03-function/rects_with_label.jpg}
        }
        \caption{図\ref{fig:rect_original}から取得した矩形領域の描画}
        \label{fig:rect_drawing}
    \end{center}
\end{figure}
図\ref{fig:rect_drawing}に示すように、矩形領域に対してランダムな色で矩形を描画し、同色で矩形左上に番号と文字列を描画する。
図\ref{fig:rect_drawing}の矩形に表示する番号は、JSONファイル内のrects\_data配列のidキーの値と一致する。
また、番号の右に表示する文字列は、JSONファイル内のrects\_data配列のlabelキーの値と一致する。
図\ref{fig:rect_drawing}の画像中にある、「単価」という文字がある矩形の左上頂点には、「0: number」と描画している。
これは、rects\_data配列のidキーに0が、labelキーにnumberがそれぞれ対応することを示す。
このように、矩形領域強調画像を出力することで、矩形領域についてJSONファイルの内容を可視化し、目視でJSONファイルの内容を確認できる。

図\ref{fig:original}内にある帳票画像内記入欄のうち、下線部の帳票画像内記入欄を切り取った画像を、図\ref{fig:underline_original}に示す。
また、図\ref{fig:underline_original}の画像に対し、下線部領域座標とラベルを取得し描画した画像を、図\ref{fig:underline_drawing}に示す。
\begin{figure}[t]
    \begin{center}
        \fbox{
            \includegraphics[width=15cm]{image/03-function/underline_original.jpg}
        }
        \caption{帳票画像にある下線部の記入欄}
        \label{fig:underline_original}
    \end{center}
\end{figure}
\begin{figure}[t]
    \begin{center}
        \fbox{
            \includegraphics[width=15cm]{image/03-function/underlines_with_label.jpg}
        }
        \caption{図\ref{fig:underline_original}の帳票画像内記入欄から取得した下線部領域の描画}
        \label{fig:underline_drawing}
    \end{center}
\end{figure}
% \begin{figure}[tp]
%     \centering
%     \fbox{
%         \includegraphics[width=15cm]{image/03-function/underline_original.jpg}
%     }
%     \caption{帳票画像にある下線部の記入欄}
%     \label{fig:underline_original}
%     \fbox{
%         \includegraphics[width=15cm]{image/03-function/underlines_with_label.jpg}
%     }
%     \caption{図\ref{fig:underline_original}の帳票画像内記入欄から取得した下線部領域の描画}
%     \label{fig:underline_drawing}
% \end{figure}
図\ref{fig:underline_drawing}に示すように、下線部領域に対して緑色で直線を描画し、同色で直線の左端点に番号と文字列を表示する。
図\ref{fig:underline_drawing}の下線部に表示する番号は、JSONファイル内のunderlines\_data配列のidキーの値と一致する。
同様に、番号の右に表示する文字列は、JSONファイル内のunderlines\_data配列のlabelキーの値と一致する。
図\ref{fig:underline_drawing}の画像中にある、「御中」という文字の左にある下線部の左端点には、「0: string」と描画している。
これは、underlines\_data配列のidキーに0が、labelキーにstringがそれぞれ対応することを示す。
このように、下線部領域強調画像を出力することで、下線部領域についてJSONファイルの内容を可視化し、目視でJSONファイルの内容を確認できる。

