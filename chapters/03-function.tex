\chapter{本提案手法の機能}\label{cha:Function}
本章では、本研究で提案する帳票画像内の記入欄を自動検出する手法の機能について説明する。

本研究では、帳票画像内の記入欄である領域の座標を取得後、文字の属性を判定し、取得文字近傍に存在する記入欄の座標に対してラベルを割り当て、領域の座標と、対応するラベルを組とするJSON形式のファイルを出力する手法を提案する。
なお、本研究において、入力である帳票画像における記入欄を帳票画像記入欄、本提案手法によって取得した座標で示す記入欄を電子フォーム記入欄、文字認識によって得た文字から推測するデータ型を属性、文字の近傍に存在する電子フォーム記入欄に対して、文字位置と属性から推測するデータ型をラベルと呼ぶ。
本研究で提案する手法は、以下の2つの機能を持つ。

\begin{itemize}
  \item 電子フォーム記入欄取得機能
  \item 電子フォーム記入欄ラベル付与機能
\end{itemize}


\section{電子フォーム記入欄取得機能}\label{sec:eform_write_space_obtainment_feature}
電子フォーム記入欄取得機能では、記入欄の対象である矩形領域および下線部領域について各領域座標を電子フォーム記入欄として取得する。
本機能は、以下の順で処理を施すことにより、電子フォーム記入欄を取得する。

\begin{enumerate}
  \item 矩形領域取得
  \item 下線部領域取得
\end{enumerate}

\subsection{矩形領域取得}\label{subsec:rect_coords_obtainment}
矩形領域取得は、矩形の領域を帳票画像記入欄とみなし、各頂点のxy座標を領域座標として取得する。

\subsection{下線部領域取得}\label{subsec:underline_coords_obtainment}
下線部領域取得は、水平な直線上の領域を帳票画像記入欄とみなし、両端点のxy座標を領域座標として取得する。


\section{電子フォーム記入欄ラベル付与機能}\label{sec:label_link}
電子フォーム記入欄ラベル付与機能は、電子フォーム記入欄として取得した領域座標に対して、ラベルを付与する。
電子フォーム記入欄にラベルを付与することにより、バリデーションチェックに必要な情報を付与することが可能である。
本機能は、以下の順で処理を施すことにより、電子フォーム記入欄にラベルを付与する。

\begin{enumerate}
  \item 文字と文字位置の取得
  \item 属性推測
  \item ラベル割付
\end{enumerate}

\subsection{文字と文字位置の取得}\label{subsec:char_and_bbox_obtainment}
入力である帳票画像に対して文字認識を行い、検出した文字と、検出した文字を囲うバウンディングボックスの各頂点の座標をそれぞれ取得文字、文字位置として取得する。

\subsection{属性推測}\label{subsec:att_prediction}
取得文字に対して、大規模言語モデルによる属性の推測を行い、日付(date)、文字列(string)、数値(number)の3つのいずれに該当するかを推測し、属性として得る。

\subsection{ラベル割付}\label{subsec:label_link}
\ref{sec:eform_write_space_obtainment_feature}節で得た領域座標と、\ref{subsec:char_and_bbox_obtainment}節で得た文字位置、\ref{subsec:att_prediction}節で得た属性を用いる。
文字位置の近傍に存在する領域座標を対象に属性を割り付け、ラベルとして得る。

ラベル割付後、領域座標と、領域座標に対応するラベルを組としたJSON形式のファイルを本提案手法の出力とする。
