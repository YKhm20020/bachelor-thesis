\chapter{おわりに}\label{cha:Conclusion}
本研究では、帳票のレイアウトを変更せず、電子フォーム作成にかかる手間と時間を削減することを目的として、記入欄検出およびラベル付与を行い、それらの内容を画像に出力するツールを試作した。
試作したツールは、以下の2つの機能を持つ。

\begin{itemize}
  \item 領域座標取得およびラベル付与機能\\
      領域座標取得およびラベル付与機能は、帳票画像中の矩形領域および下線部領域について、領域座標を電子フォーム記入欄として取得し、それぞれにラベルを割り付ける機能である。
      ラベルを割り付けることにより、バリデーションチェックに必要な情報を付与することができる。
  \item 領域強調画像出力機能\\
      領域強調画像出力機能は、入力である帳票画像に対して、取得した領域座標とラベルを描画することによって、強調表示したPNG画像を出力する機能である。
      これによって、領域座標取得およびラベル付与機能で出力したJSONファイルの内容を、目視で確認しやすくなる。
      本機能で出力する画像は、矩形領域を強調した矩形領域強調画像と、下線部領域を強調した下線部領域強調画像の計2枚である。
\end{itemize}

適用例を用いて、試作したツールが領域座標の取得と、ラベル割付が一部を除いて正常に動作することを確認した。
なお、正常に動作しない原因は、文字認識が失敗したことによって、想定とは異なるラベルを領域座標に割り付けたことや、記入する内容を示す文字が、帳票画像記入欄の右にあることによって、領域座標の右にある文字の属性を参照しないためであることを確認した。
また、JSONファイルの内容と、2枚の領域強調画像が正しいことと、それらが対応した状態で、帳票画像のレイアウトに忠実に出力することを確認した。

試作したツールの有用性を評価するため、電子フォーム記入欄を配置するまでにかかる時間を計測する実験を行った。
2枚の帳票画像について、試作したツールを適用することによって、GUIツールのみを用いた場合と比較して、それぞれ平均で4分46秒(約54.17\%)、2分42秒(約37.41\%)短いことから、電子フォーム記入欄を配置するまでにかかる時間を削減できることを確認した。
また、試作ツールの出力である領域座標とラベルの精度を確認するため、それぞれの適合率と再現率を評価した。
領域座標およびラベルの再現率は、適合率に比べ高いことを確認した。
これは、電子フォーム記入欄の削除による修正回数が、配置による修正回数と比較して多いことを示すため、電子フォーム作成にかかる手間を削減できることを確認した。

以上のことから、本研究で試作したツールは、電子フォーム作成にかかる手間と時間の削減に有用であると言える。

今後の課題を、以下に示す。

\begin{itemize}
    \item 帳票画像記入欄と記入内容を示す欄の区別\\
        特に矩形が隣接する形式の帳票である場合、行や列の先頭に、記入内容を示す矩形の欄があるときがある。
        これにより、帳票画像記入欄ではない矩形の領域座標を不要に取得してしまうため、対応する必要がある。
    \item 領域座標に割り付けるラベルの安定\\
        領域座標に割り付けるラベルは、文字認識と大規模言語モデルの出力に依存する。
        現在のTesseract-OCRでは、文字を100\%正確に認識することができず、大規模言語モデルの出力は実行の度に変化する。
        これにより、電子フォーム記入欄に想定とは異なるラベルを割り付けてしまう場合があるため、対応する必要がある。
    \item レイアウトによる精度の低下\\
        背景や帳票画像記入欄の一部に色が付いている、押印があるなどの二値画像は、黒の画素の数が増加する。
        これにより、帳票画像記入欄の検出やラベル割付の精度が低下してしまう場合があるため、対応する必要がある。
    \item 記入内容を示す文字の判定\\
        ある帳票画像記入欄について、記入する内容を示す文字が右にある場合、ラベルを更新できない。
        これにより、想定とは異なるラベルを割り付けてしまう場合があるため、対応する必要がある。
    \item 帳票画像記入欄を文字で示す場合の領域座標の取得\\
        文字のみで帳票画像記入欄であることを示す場合は、領域座標を取得できない。
        これにより、一部の帳票画像記入欄を認識できないため、対応する必要がある。
    \item 撮影環境による精度の低下\\
        撮影する環境によって、大津の二値化で取得する閾値が変化し、うまく二値化することができない。
        これにより、矩形と下線部を取得できない場合や、文字認識が失敗する場合があるため、対応する必要がある。
\end{itemize}