\chapter{おわりに}\label{cha:Conclusion}
本研究では、電子フォーム作成時間の短縮を目的として、帳票画像内における記入欄の領域座標、および、その記入欄に入力する内容のデータ型をまとめたJSONファイルに加えて、検知した記入欄を強調表示した画像2枚を出力する、記入欄の検出およびラベル付与手法を提案した。
本提案手法では、帳票画像を入力として、領域座標とラベルをまとめたJSONファイルと、JSONファイルの内容に対応し、領域とラベルを帳票画像に描画した2枚の領域強調画像を出力とするツールの作成を手段とした。

適用例を用いて、本ツールが領域座標の取得と、ラベル割付が一部を除いて正常に動作することを確認した。
なお、正常に動作しない原因は、文字認識が失敗したことによって、想定とは異なるラベルを領域座標に割り付けたことや、記入する内容を示す文字が、帳票画像記入欄の右にあることによって、領域座標の右にある文字の属性を参照しないためであることを確認した。
また、JSONファイルの内容と、2枚の領域強調画像が正しく対応した状態で出力することを確認した。

本ツールの有用性を評価するため、実験を行い、電子フォーム記入欄を配置するまでにかかる時間を削減できたことを確認した。
また、本ツールの出力について、精度を確認するため、領域座標とラベルの、それぞれの適合率と再現率を評価し、削減した時間に対して、修正する必要がある電子フォーム記入欄の割合が妥当であることを確認した。

以上のことから、本研究で提案した手法は、電子フォーム作成にかかる手間と時間の削減に有用であると言える。

今後の課題を、以下に示す。

\begin{itemize}
    \item 記入欄と記入内容を示す欄の区別\\
        特に矩形が隣接する形式の帳票である場合、行や列の先頭に、記入内容を示す矩形の欄があるときがある。
        これにより、帳票画像記入欄ではない矩形の領域座標を不要に取得してしまうため、対応する必要がある。
    \item 領域座標に割り付けるラベルの安定\\
        領域座標に割り付けるラベルは、文字認識と大規模言語モデルの出力に依存する。
        現在のTesseract-OCRでは、文字を100\%正確に認識することができず、大規模言語モデルの出力は実行の度に変化する。
        これにより、電子フォーム記入欄に想定とは異なるラベルを割り付けてしまう場合があるため、対応する必要がある。
    \item レイアウトによる精度の低下\\
        背景や帳票画像記入欄の一部に色が付いている、押印があるなどの二値画像は、黒の画素の数が増加する。
        これにより、帳票画像記入欄の検出やラベル割付の精度が低下してしまう場合があるため、対応する必要がある。
    \item 記入内容を示す文字の判定\\
        ある帳票画像記入欄について、記入する内容を示す文字が右にある場合、ラベルを更新することができない。
        これにより、想定とは異なるラベルを割り付けてしまう場合があるため、対応する必要がある。
\end{itemize}