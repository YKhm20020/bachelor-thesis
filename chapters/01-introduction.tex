\chapter{はじめに}\label{cha:Introduction}
2019年4月に電子帳簿保存法が改正され、帳簿書類の電子データ保存が義務付けられたことにより、帳簿書類の電子化が推進されている\cite{電子帳簿保存法}。
また、総務省の令和2年版「情報通信白書」では、60.4\%の企業が社内業務のペーパーレス化に取り組んでいると答えた\cite{デジタルデータの経済的価値の計測と活用の現状に関する調査研究}。
しかし、Biz Clipによる文書管理実態調査2023によると、紙の文書が介在する業務工程のうち、契約・申請書類は、2020年時点で2023年時点で66.5\%、2023年で59.1\%であるという結果となり、特に契約書や請求書といった帳票は、比較的に紙媒体が多いことがわかった\cite{文書管理実態調査2023}。

帳票の電子化は、スキャナや写真などで帳票を撮影することで実現できる。
しかし、帳票の記入内容の確認は、人が目視で確認する必要がある。
効率的に記入内容を管理するためには、その記入内容をデータとして保存する必要がある。
その方法の1つとして、電子フォームを用いる方法がある。
電子フォームとは、従来紙の帳票で行っていた申請などの業務を電子化し、アプリケーションを介して入力した項目をそのまま業務用のデータとして利用するための仕組みである\cite{電子フォーム}。
電子フォームを用いることで、記入内容をデータとして保存することができる。
しかし、電子フォームの作成するには、以下の2つの課題がある。

\begin{itemize}
  \item 電子フォームの作成に時間がかかる
  \item 使い慣れた既存の帳票のレイアウトが変更となる場合がある
\end{itemize}

これらの課題を解決するため、電子フォームを自動作成するツールやサービスがある。
既存の電子フォーム作成ツール\cite{i-Reporter}\cite{Create!Form}は、Excelファイルを入力とすることにより、記入欄の位置を取得することができ、セルに設定した書式設定を保持した状態で、計算式や電子フォームを自動で作成できる。
しかし、Excelファイルを入力とした場合は、レイアウトが変更となる可能性がある。
また、画像ファイルを入力とした場合は、入力する画像を背景とするため、レイアウトは変更とならないが、マウス操作で記入欄を配置する必要があるため、手間と時間がかかる。
さらに、書式についての情報がないため、記入欄に書式の設定をすることができない。

そこで本研究は、前述した2つの課題点を同時に解決するため、帳票画像内における記入欄の領域座標、および、その記入欄に入力する内容のデータ型をまとめたJSONファイルに加えて、検知した記入欄を強調表示した画像2枚を出力する、ラベル付き領域座標出力手法を提案する。
まず、帳票画像内にある記入欄の位置を取得するために、記入欄を画像処理によって検出する。
次に、検出した記入欄に入力する内容のデータ型を決定するために、入力内容を示す文字を認識し、大規模言語モデルを用いて、記入する内容が、日付(date)、文字列(string)、数値(number)の3種類のうちいずれかを、ラベルとして割り付ける。
そして、取得した記入欄の領域座標を参照して、帳票画像に色を付けて描画することで、取得した記入欄を強調表示した画像を出力する。
記入欄の検出およびラベル付与手法の提案を実現するための手段として、これらの処理を行うツールを実装する。

本論文の構成を、以下に示す。\\
第2章では、本研究に必要となる前提知識について説明する。\\
第3章では、本ツールの持つ機能について説明する。\\
第4章では、本ツールの実装について説明する。\\
第5章では、本ツールの適用例について説明する。\\
第6章では、本ツールについて考察する。\\
第7章では、本研究のまとめと今後の課題を示す。
