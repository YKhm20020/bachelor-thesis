\chapter{はじめに}\label{cha:Introduction}
2019年4月に電子帳簿保存法が改正され、帳簿書類の電子データ保存が義務付けられたことにより、帳簿書類の電子化が推進されている\cite{電子帳簿保存法}。
また、総務省の令和2年版「情報通信白書」では、60.4\%の企業が社内業務のペーパーレス化に取り組んでいると答えた\cite{デジタルデータの経済的価値の計測と活用の現状に関する調査研究}。
しかし、Biz Clip\cite{Biz Clip}による文書管理実態調査2023によると、紙の文書が介在する業務工程のうち、契約・申請書類は、2020年時点で2023年時点で66.5\%、2023年で59.1\%であるという結果となり、特に契約書や請求書といった帳票\cite{帳票}は、比較的に紙媒体が多いことがわかった。

帳票の電子化は、スキャナや写真などで帳票を撮影することで実現できるが、記入内容の確認は、人が目視で確認する必要がある。
効率的に記入内容を管理するためには、データとして保存する必要があるが、電子フォームを作成するにあたり、以下の課題がある。

\begin{itemize}
  \item 電子フォームの作成と導入に時間がかかる
  \item 使い慣れた既存のレイアウトが変更となる場合がある
\end{itemize}

これらの課題を解決するため、電子フォーム作成ツールがある。
既存の電子フォーム作成ツール\cite{Create!Form}\cite{i-Reporter}は、Excelファイルを入力とすることにより、記入欄の位置を取得することができる。
一方で、画像ファイルに対しては入力する画像を背景として、人手による作業で記入欄を配置する必用があるため、手間と時間がかかる。

そこで、本研究は、帳票の画像に対して記入欄を配置するには時間がかかるという課題点を解決するため、帳票画像内における記入欄を検出し、xy座標を取得する。
さらに、記入欄に記載すべき内容を文字認識し、大規模言語モデルによる推測を行うことによって、記入欄の座標に対して、日付(date)、文字列(string)、数値(number)の3種類のうちいずれかをラベルとして割り付ける。
これによって、帳票画像内の記入欄を検出し、ラベルを付与する手法を提案する。

本論文の構成を、以下に示す。\\
第2章では、本研究に必要となる前提知識について説明する。\\
第3章では、本提案手法の持つ機能について説明する。\\
第4章では、本提案手法の実装について説明する。\\
第5章では、本提案手法の適用例について説明する。\\
第6章では、本提案手法について考察する。\\
第7章では、本研究のまとめと今後の課題を示す。
