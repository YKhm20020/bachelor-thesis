\chapter{はじめに}\label{cha:Introduction}
2019年4月に電子帳簿保存法が改正され、帳簿書類の電子データ保存が義務付けられたことにより、帳簿書類の電子化が推進されている\cite{電子帳簿保存法}。
また、総務省の令和2年版「情報通信白書」では、60.4\%の企業が社内業務のペーパーレス化に取り組んでいると答えた\cite{デジタルデータの経済的価値の計測と活用の現状に関する調査研究}。
しかし、Biz Clipの文書管理実態調査2023によると、紙の文書が介在する業務工程のうち、契約・申請書類は、2020年時点で2023年時点で66.5\%、2023年時点で59.1\%であり、特に契約書や請求書といった帳票は、比較的に紙媒体が多いことがわかった\cite{文書管理実態調査2023}。

帳票の電子化は、スキャナや写真などで帳票を撮影することで実現できる。
しかし、帳票に記入した内容は、人が目視で確認する必要がある。
効率的に記入内容を管理するためには、その記入内容をデータとして保存する必要がある。
その方法の1つとして、電子フォームを用いる方法がある。
電子フォームとは、従来紙の帳票で行っていた申請などの業務に用いる帳票を電子化し、アプリケーションを介して入力した項目をそのまま業務用のデータとして利用するための仕組みである\cite{電子フォーム}。
電子フォームを用いることによって、記入内容をデータとして保存できる。
しかし、電子フォームを作成するには、以下の2つの課題がある。

\begin{itemize}
    \item 電子フォームの作成に時間がかかる
    \item 使い慣れた既存の帳票のレイアウトが変わる場合がある
\end{itemize}

これらの課題を解決するため、電子フォームを自動作成するツールやサービスがある。
既存の電子フォーム作成ツールであるi-Reporter\cite{i-Reporter}やCreate!Form\cite{Create!Form}は、Excelファイルを入力とすることにより、記入欄の位置を取得することができ、セルに設定した書式設定を保持した状態で、電子フォームを自動で作成できる。
しかし、Excelファイルを入力とした場合は、レイアウトが変わる可能性がある。
また、画像ファイルを入力とした場合は、入力する画像を背景とするため、レイアウトは変わらないが、GUIツールを用いたマウス操作で記入欄を配置する必要があるため、手間と時間がかかる。
さらに、書式についての情報がないため、記入欄に書式の設定をすることができない。

そこで本研究は、電子フォームの作成にあたって、使い慣れたレイアウトを変更せず、作成に必要な手間と時間を削減することを目的として、帳票画像内における記入欄の領域座標とその記入欄に入力する内容のデータ型をまとめたJSONファイルに加えて、検知した記入欄を強調表示した画像2枚を出力する、記入欄の検出およびラベル付与手法を提案する。
まず、帳票画像内にある記入欄の位置を取得するために、記入欄を画像処理によって検出する。
次に、検出した記入欄に入力する内容のデータ型を決定するために、入力内容を示す文字を認識し、大規模言語モデルを用いて、記入する内容が、日付(date)、文字列(string)、数値(number)の3種類のうちいずれかを、ラベルとして割り付ける。
そして、取得した記入欄の領域座標を参照して、帳票画像に色を付けて描画することで、取得した記入欄を強調表示した画像を出力する。
本研究では、提案する記入欄の検出およびラベル付与手法を試行するための手段として、これらの処理を行うツールを試作する。
試作するツールは、以下の2つの機能を持つ。

\begin{itemize}
    \item 領域座標取得およびラベル付与機能\\
        領域座標取得およびラベル付与機能は、帳票画像中の矩形領域および下線部領域について、領域座標を電子フォーム記入欄として取得し、それぞれにラベルを割り付ける機能である。
        ラベルを割り付けることにより、バリデーションチェック\cite{バリデーションチェック}に必要な情報を付与することができる。
    \item 領域強調画像出力機能\\
        領域強調画像出力機能は、入力である帳票画像に対して、取得した領域座標とラベルを描画することによって、強調表示したPNG画像を出力する機能である。
        これによって、領域座標取得およびラベル付与機能で出力したJSONファイルの内容を、目視で確認しやすくなる。
        本機能で出力する画像は、矩形領域を強調した矩形領域強調画像と、下線部領域を強調した下線部領域強調画像の計2枚である。
\end{itemize}

本論文の構成を、以下に示す。\\
第2章では、本研究に必要となる前提知識について説明する。\\
第3章では、試作したツールの持つ機能について説明する。\\
第4章では、試作したツールの実装について説明する。\\
第5章では、試作したツールの適用例について説明する。\\
第6章では、試作したツールについて考察する。\\
第7章では、本研究のまとめと今後の課題を示す。