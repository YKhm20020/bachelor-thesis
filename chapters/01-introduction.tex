\chapter{はじめに}\label{cha:Introduction}
(簡単なメモ。パワーポイントとほぼ同じ内容なので、あとで詳しく背景を書くこと)

既存の電子フォーム作成ツールは、Excelファイルや入力とすることにより、記入欄の位置を取得することができる。一方で、画像に対しては入力する画像を背景として、人手による作業で記入欄を配置している。
本研究は、帳票画像に対して記入欄を配置するには時間がかかるという問題点を解決するため、帳票画像内における記入欄を検出し、xy座標を取得する。
さらに、記入欄に記載すべき内容を文字認識し、大規模言語モデルによる推測を行うことによって、取得文字近傍に存在する記入欄の座標に対して、日付(date)、文字列(string)、数値(number)の3種類のうちいずれかをラベルとして割り付ける。

本論文の構成を、以下に示す。\\
  (各章についての説明をあとで詳しく書くこと)
