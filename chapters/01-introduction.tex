\chapter{はじめに}\label{cha:Introduction}
2019年4月に電子帳簿保存法が改正され、帳票のデータ保存が義務付けられたことにより、帳票の電子化が推進されている\cite{電子帳簿保存法}。
また、総務省の令和2年版「情報通信白書」では、60.4\%の企業が社内業務の電子化に取り組んでいると答えている\cite{デジタルデータの経済的価値の計測と活用の現状に関する調査研究}。
Biz Clipの文書管理実態調査2023によると、紙の文書が介在する業務工程のうち、契約・申請書類は、2020年時点で66.5\%、2023年時点で59.1\%であり、特に契約書や請求書といった帳票は、紙媒体が比較的多いことがわかる\cite{文書管理実態調査2023}。
また、キーマンズネットの帳票類のデジタル化の状況(2022年)によると、企業ごとの帳票の管理形式は、44.9\%が部分的にデータ化しているが、紙に出力して管理しており、5.8\%が全て紙で運用し、紙で管理している\cite{帳票類のデジタル化の状況}。
この調査結果から、50.7\%の企業が、帳票の管理形式に紙を用いていることがわかる。
これらの調査より、帳票の電子化が推進されているものの、未だに約半数の企業は管理形式に紙を採用していることがわかる。

帳票の電子化は、スキャナやカメラなどで帳票を撮影することで実現できる。
簡単に実現できるというメリットがある一方で、帳票に記入した内容は、人が目視で確認する必要があるというデメリットがある。
効率的に記入内容を管理するためには、その記入内容をデータとして保存する必要がある。
その方法の1つとして、電子フォームを用いる方法がある。
電子フォームとは、従来紙の帳票で行っていた申請などの業務に用いる帳票を電子化し、アプリケーションを介して入力した項目をそのまま業務用のデータとして利用するための仕組みである\cite{電子フォーム}。
電子フォームを自動作成するツールやサービスがあり、既存の電子フォーム作成ツールである、i-Reporter\cite{i-Reporter}やCreate!Form\cite{Create!Form}は、帳票をデザインしたExcelファイルと、画像ファイルを入力として、電子フォームを作成できる。
帳票をデザインしたExcelファイルを入力とすることにより、記入欄の位置を取得することができ、セルに設定した書式設定を保持した状態で、電子フォームを自動で作成できる。
しかし、これらの電子フォームは、帳票をExcelファイルとして管理

しかし、紙媒体の帳票から電子フォームへ移行するには、以下の2つの課題がある。

\begin{itemize}
  \item 電子フォームの作成に時間がかかる
  \item 使い慣れた既存の帳票のレイアウトが変わる場合がある
\end{itemize}

電子フォームの作成に時間がかかる問題を解決するため、
しかし、紙媒体のレイアウトをもとにExcelでレイアウトを書き直す必要があり、レイアウトが変わる場合がある課題を解決できない。
また、帳票の画像ファイルを入力とした場合は、入力する帳票画像を背景とするため、レイアウトは変わらない。
しかし、GUIツールを用いたマウス操作で記入欄を配置する必要があるため、電子フォームの作成に時間がかかる課題を解決できない。
さらに、書式についての情報がないため、記入欄に書式の設定をすることができない。

そこで本研究は、帳票のレイアウトを変更せず、電子フォーム作成にかかる手間と時間を削減することを目的として、記入欄およびラベル付与ツールを試作する。

% 帳票画像を入力として、  帳票画像内における記入欄の領域座標とその記入欄に入力する内容のデータ型をまとめたJSONファイルに加えて、検知した記入欄を強調表示した画像2枚を出力する、
% まず、帳票画像内にある記入欄の位置を取得するために、記入欄を画像処理によって検出する。
% 次に、検出した記入欄に入力する内容のデータ型を決定するために、帳票画像から入力内容を示す文字を認識し、大規模言語モデルを用いて、記入する内容が、日付(date)、文字列(string)、数値(number)の3種類のうちいずれかを、ラベルとして割り付ける。
% 最後に、取得した記入欄の領域座標を参照して、帳票画像に色を付けて描画することで、取得した記入欄を強調表示した画像を出力する。


試作したツールは、以下の2つの機能を持つ。

\begin{itemize}
  \item 領域座標取得およびラベル付与機能\\
      領域座標取得およびラベル付与機能は、帳票画像中の矩形領域および下線部領域について、領域座標を電子フォーム記入欄として取得し、それぞれにラベルを割り付ける機能である。
      ラベルを割り付けることにより、バリデーションチェックに必要な情報を付与することができる。
  \item 領域強調画像出力機能\\
      領域強調画像出力機能は、入力である帳票画像に対して、取得した領域座標とラベルを描画することによって、強調表示したPNG画像を出力する機能である。
      これによって、領域座標取得およびラベル付与機能で出力したJSONファイルの内容を、目視で確認しやすくなる。
      本機能で出力する画像は、矩形領域を強調した矩形領域強調画像と、下線部領域を強調した下線部領域強調画像の計2枚である。
\end{itemize}

本論文の構成を、以下に示す。\\
第2章では、本研究に必要となる前提知識について説明する。\\
第3章では、試作したツールの機能について説明する。\\
第4章では、試作したツールの実装について説明する。\\
第5章では、試作したツールの適用例について説明する。\\
第6章では、試作したツールについて考察する。\\
第7章では、本研究のまとめと今後の課題を示す。