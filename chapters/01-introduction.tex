\chapter{はじめに}\label{cha:Introduction}
2019年4月に電子帳簿保存法が改正され、帳票のデータ保存が義務付けられたことにより、帳票の電子化が推進されている\cite{電子帳簿保存法}。
また、総務省の令和2年版「情報通信白書」では、60.4\%の企業が社内業務の電子化に取り組んでいると答えている\cite{デジタルデータの経済的価値の計測と活用の現状に関する調査研究}。
キーマンズネットの帳票類のデジタル化の状況(2022年)によると、企業ごとの帳票の管理形式は、44.9\%が部分的にデータ化しているが、紙に出力して管理しており、5.8\%が全て紙で運用し、管理している\cite{帳票類のデジタル化の状況}。
この調査結果から、50.7\%の企業が、帳票の管理形式に紙を用いていることがわかる。
これらの調査より、帳票を電子化する動きがある一方で、約半数の企業は管理形式に紙を採用していることがわかる。

帳票の電子化は、スキャナやカメラで帳票を撮影することによって実現できる。
簡単に電子化できるというメリットがある一方で、帳票に記入した内容は、人が目視で確認する必要があるというデメリットがある。
効率的に記入内容を管理するためには、その記入内容をデータとして保存する必要がある。
その方法の1つとして、電子フォームを用いる方法がある。
電子フォームとは、従来紙の帳票で行っていた申請などの業務に用いる帳票を電子化し、アプリケーションを介して入力した項目をそのまま業務用のデータとして利用するための仕組みである\cite{電子フォーム}。

電子フォームを自動で作成するために、複数のツールが開発されている。
既存の電子フォーム作成ツールである、i-Reporter\cite{i-Reporter}やCreate!Form\cite{Create!Form}は、帳票のExcelファイル、もしくは、画像ファイルを入力として、電子フォームを作成できる。
これらの既存の電子フォーム作成ツールは、帳票をExcelファイルとして管理している場合、Excelファイルのレイアウトとセルの書式を保ったまま電子フォームを簡単に作成できる。
一方、帳票を紙媒体で管理している場合は、既存の電子フォーム作成ツールを使用するために、帳票をExcelファイルとして作成し、入力とするか、紙媒体の帳票を撮影した画像ファイルを入力として、電子フォームを手作業で作成する必要があり、それぞれ以下の2つの課題がある。

\begin{itemize}
  \item Excelファイルを入力とする場合、紙媒体の帳票のレイアウトをもとにExcelファイルを新たに作成する必要があり、使い慣れた既存の帳票のレイアウトが変わる場合がある。
  \item 画像ファイルを入力とする場合、帳票の画像を背景として、GUIツールを用いたマウス操作で記入欄を配置する必要があるため、電子フォームの作成に時間がかかる。
\end{itemize}

そこで本研究は、帳票のレイアウトを保持したまま変更せず、電子フォーム作成にかかる時間を削減することを目的として、記入欄自動検出およびラベル割付機能を持つツールを試作する。

% 帳票画像を入力として、  帳票画像内における記入欄の領域座標とその記入欄に入力する内容のデータ型をまとめたJSONファイルに加えて、検知した記入欄を強調表示した画像2枚を出力する、
% まず、帳票画像内にある記入欄の位置を取得するために、記入欄を画像処理によって検出する。
% 次に、検出した記入欄に入力する内容のデータ型を決定するために、帳票画像から入力内容を示す文字を認識し、大規模言語モデルを用いて、記入する内容が、日付(date)、文字列(string)、数値(number)の3種類のうちいずれかを、ラベルとして割り付ける。
% 最後に、取得した記入欄の領域座標を参照して、帳票画像に色を付けて描画することで、取得した記入欄を強調表示した画像を出力する。

試作したツールは、以下の2つの機能を持つ。

\begin{itemize}
  \item 領域座標自動取得およびラベル割付機能\\
      領域座標自動取得およびラベル割付機能は、帳票画像を入力として、帳票画像中の矩形領域および下線部領域について、領域座標を電子フォーム内記入欄として自動で取得し、各領域座標とそれに対してラベルを割り付けた結果をまとめたJSONファイルを出力とする機能である。
      ラベルを割り付けることで、バリデーションチェックに必要な情報を付与できる。
  \item 領域強調画像出力機能\\
      領域強調画像出力機能は、入力である帳票画像に対して、取得した領域座標とラベルを描画することによって、強調表示したPNG画像を出力する機能である。
      これによって、領域座標自動取得およびラベル割付機能で出力したJSONファイルの内容を、目視で確認しやすくする。
      本機能で出力する画像は、矩形領域を強調した矩形領域強調画像と、下線部領域を強調した下線部領域強調画像の計2枚である。
\end{itemize}

本論文の構成を、以下に示す。\par
第2章では、本研究に必要となる前提知識について説明する。\par
第3章では、試作したツールの機能について説明する。\par
第4章では、試作したツールの実装について説明する。\par
第5章では、試作したツールの適用例について説明する。\par
第6章では、試作したツールについて考察する。\par
第7章では、本研究のまとめと今後の課題を示す。