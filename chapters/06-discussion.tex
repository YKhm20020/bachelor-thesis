\chapter{考察}\label{cha:Discussion}
本研究では、電子フォーム作成時間の削減を目的としたラベル付き記入欄検出手法の提案を行った。
本提案手法は、以下に示す2つの機能を持つ。

\begin{itemize}
  \item ラベル付き領域座標取得機能
  \item 領域描画画像出力機能
\end{itemize}

5章では、本提案手法が持つ2つの機能について、正しく動作することを確認した。
本章では、まず、本提案手法の有用性について考察する。次に、本提案手法と関連研究を比較する。最後に、本提案手法の問題点について述べる。

\section{本提案手法の有用性に関する評価}\label{sec:evalue_usefulness}
本提案手法の有用性を評価するため、本提案手法を、電子フォーム作成ツールであるPhotolizeに適用し、実験を行う。
Photolizeは、スマートフォンで撮影した帳票の画像に対して、電子フォーム記入欄を配置することによって、電子フォームを作成するサービスである\cite{Photolize}。
本提案手法をPhotolizeに適用することで、帳票画像記入欄を検出し、領域座標とラベルをまとめたJSONファイルから、電子フォーム記入欄を配置する手間と時間を削減することができる。

実験は、2枚の帳票画像に対して本提案手法を適用し、検出できなかった帳票画像記入欄や、誤検出によって電子フォーム記入欄に対して、被験者には、Photolizeを利用し、電子フォーム記入欄を配置してもらう。
実験の対象とした帳票の画像を、図Aと図Bに示す。
図Aは、本提案手法が入力の対象とする画像のうち、電子文書の画像であり、図Bは、本提案手法が入力の対象とする画像のうち、電子化文書の画像である。

実験は、宮崎大学工学部情報システム工学科に所属する学部4年生X名、および修士1年生Y名、2年生Z名の計A名を対象として行った。

実験方法は、ケースAとケースYに分けて行う。
ケースAでは、図Aに対して、本提案手法を適用せず、手作業で電子フォーム記入欄を配置してもらう。
次に、図Bに対して、本提案手法を適用し、修正対象の電子フォーム記入欄を修正し、電子フォーム記入欄を配置してもらう。
ケースBでは、図Aに対して、本提案手法を適用し、修正対象の電子フォーム記入欄を修正し、電子フォーム記入欄を配置してもらう。
次に、図Bに対して、本提案手法を適用せず、手作業で電子フォーム記入欄を配置してもらう。
なお、修正対象の電子フォーム記入欄に該当する電子フォーム記入欄を、以下に示す。

\begin{itemize}
  \item 誤検出によって、本来帳票画像内には帳票記入欄が存在しない場所に配置した、電子フォーム記入欄。
  \item 帳票画像内に存在する帳票記入欄を検出できず、配置できていない電子フォーム記入欄。
  \item 本来想定するラベルとは異なるラベルを割り付けた、電子フォーム記入欄
\end{itemize}

被験者X名を2つのグループに分け、片方のグループにケースAの実験を行い、もう片方のグループにケースBの実験を行う。


実験で計測する時間について、以下の名称を用いる。

\begin{itemize}
  \item 実行時間
        本提案手法を適用し、JSONファイルと2枚の領域強調画像を出力するまでの時間。
  \item 配置完了時間
        電子フォーム記入欄の配置が完了するまでの時間。
  \item 修正完了時間
        配置した電子フォーム記入欄を確認し、修正対象の電子フォーム記入欄があった場合、それを修正するまでの時間。
\end{itemize}

実験の手順を、以下に示す。

\begin{enumerate}
  \item 開始前に、対象の帳票画像とPhotolizeについて説明し、Photolizeにおける電子フォーム記入欄の配置に必要な操作も併せて説明する。
  \item 帳票画像に対して、本提案手法を適用し、実行時間を計測する。
  \item 配置完了時間を計測する。
  \item 修正完了時間を計測する。
\end{enumerate}

本提案手法を適用せず、手作業で電子フォーム記入欄を配置



図A、および図Bの画像に対して、本提案手法を適用し、出力した矩形領域強調画像と下線部領域強調画像を、それぞれ図A1、図A2に示す。
同様に、電子化文書の帳票画像である図Bに対して、本提案手法を適用し、出力した矩形領域強調画像と下線部領域強調画像を、それぞれ図B1、図B2に示す。
配置するべき電子フォーム記入欄の数は、図Aは、矩形領域が個、下線部領域が個の計個であり、図Bは、矩形領域が個、下線部領域が個の計個である。

\subsection{電子フォーム記入欄の配置完了までにかかる時間に関する評価}\label{subsec:evalue_required_time}

\subsection{配置した電子フォーム記入欄の精度に関する評価}\label{subsec:evalue_accuracy}
理想は、手作業で配置した精度の100\%に近い精度が提案手法適用時にも出ること。



\section{関連研究}\label{sec:relation_research}



\section{本提案手法の問題点}\label{sec:AWSEL_problems}