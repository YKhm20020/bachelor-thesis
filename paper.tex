%%
% 今時jarticleやjbook使ってる人いる?時代はjsarticleかjsbookだよ
% ついでに言うと、uplatexってのはplatexの上位互換、これを使わないなんて旧世代だよね
%
\documentclass[uplatex, report, a4j, 10pt]{jsbook}

\usepackage{packages/miyazaki-u-paper}   % 宮崎大学工学部の卒論の基本(片山先生作)を、僕がちょっと書き換えちゃった(テヘッ
\usepackage{enumitem}           % enumerate?古い古い
\usepackage[dvipdfmx]{graphicx} % 当然dvipdfmなんて使ってないよね
\usepackage[dvipdfmx]{color}    % listingsを使うときにはこれも必須、dvipdfmxを変えちゃうとgraphicxのdvipdfmxも変わるよ
\usepackage{listings, packages/jlisting} % コードを埋め込むなら必須
\usepackage{txfonts}            % フォントといえばやっぱりtxfonts、今はnewtxってのもあるらしい
\usepackage{verbatim}           % コメントアウトしてくれる便利なプリアンブルが使える \begin{comment} ... \end{comment}
\usepackage{url}
% \usepackage{easy-todo}
\usepackage[hdivide={21mm, , 21mm}, vdivide={30mm, , 25mm}]{geometry} % スタイルを少し変えたくても\hoffset, \voffsetは使わないでね
\usepackage{multirow}
\usepackage{ascmac}
\usepackage{subcaption}
\usepackage{colortbl}
\usepackage[dvipdfmx, hidelinks]{hyperref} % リンクを付けてくれる。
\usepackage{pxjahyper}          % リンクを付けてくれる(日本語)

% ソースコードの設定
\lstset{
  breaklines = true,%自動で折り返す
  basicstyle={\footnotesize\ttfamily},
  numberstyle={\scriptsize},
  stepnumber=1,
  numbersep=1zw,
  lineskip=-0.5ex,
  frame=single,
  numbers=left,%行番号を左に
  framexleftmargin=6mm,%行番号をフレーム内に
  numberstyle=\scriptsize,%行番号のサイズ
  stepnumber=1%1行おきに行番号を
}

% \addtolength\oddsidemargin{-1.5zw}  % 魔法の呪文01
% \addtolength\evensidemargin{-1.5zw} % 魔法の呪文02
% \addtolength\textwidth{3.0zw}       % 魔法の呪文03

\setcounter{page}{1}
\newcommand{\ttt}[1]{\texttt{#1}}
\newcommand{\ift}{if-then-else}
\newcommand{\toolName}{AWSETIL}  % ツール名を設定

%%
% miyazaki-u-paper.sty用設定値
%
% \degree{g} % Graduateのg or Masterのm
% \figurenumbering{f} % 図目次を付ける場合はt (真) を持つ真偽値を引数に取る関数
% \tablenumbering{f} % 表目次を付ける場合はt (真) を持つ真偽値を引数に取る関数
\title{電子フォーム作成時間の削減を目的とした\\帳票画像の書類記入欄の自動検出ツール\toolName の試作\\}
\author{木村 優哉}
\nendo{5} % 年度
\advisor{片山 徹郎 教授} % 修論では無視する
\major{情報システム工学科}


\begin{document}
\maketitle

% \setlength\textfloatsep{0pt}% 魔法の呪文04

%
% 概要
%
\preface{概要}
作成予定
Auto Write-in Space Extract Tool in Image with Labels (AWSETIL)

%
% 本文
%
\chapter{はじめに}\label{cha:Introduction}
(簡単なメモ。パワーポイントとほぼ同じ内容なので、あとで詳しく背景を書くこと)

既存の電子フォーム作成ツールである、Create!Form\cite{Create!Form}や、i-Reporter\cite{i-Reporter}などのツールは、Excelファイルを入力とすることにより、記入欄の位置を取得することができる。
一方で、帳票の画像に対しては入力する画像を背景として、人手による作業で記入欄を配置している。

そこで、本研究は、帳票の画像に対して記入欄を配置するには時間がかかるという問題点を解決するため、帳票画像内における記入欄を検出し、xy座標を取得する。
さらに、記入欄に記載すべき内容を文字認識し、大規模言語モデルによる推測を行うことによって、取得文字近傍に存在する記入欄の座標に対して、日付(date)、文字列(string)、数値(number)の3種類のうちいずれかをラベルとして割り付ける。

本論文の構成を、以下に示す。\\
  (各章についての説明をあとで詳しく書くこと)

\chapter{研究の準備}\label{cha:Preparation}
(簡単なメモ。入力画像についてのみ明記。あとで詳しく書くこと)

本章では、本研究に必要な前提知識について説明する。


\section{入力対象の画像}\label{sec:input_images}
本提案手法の入力対象である画像については、以下の条件を全て満たす画像とする。
なお、帳票は帳簿と伝票の総称であり\cite{帳票}、電子文書は、Wordやテキストファイルなど、デジタル情報として作成した文書を指し、電子化文書は紙媒体の書類をスキャンしたPDFファイルや画像などとして保存した文書を指す\cite{電子文書と電子化文書}。
電子化文書画像については、本提案手法においては、スマートフォンのカメラで撮影した帳票の画像を想定している。

\begin{itemize}
  \item 帳票の電子文書または電子化文書の画像である。
  \item 日本語かつ横書きの帳票である。
  \item 文字が手書きではない。
  \item 入力欄が矩形または下線で示されている。
\end{itemize}

なお、入力対象とする画像の座標系は、画像左上隅画素を原点(0,0)とし、右方向をX軸の正の向き、下方向をY軸の正の向きと定義する。

\section{バリデーションチェック}\label{sec:validation_check}


\section{OpenCV}\label{sec:OpenCV}
OpenCV(Open Source Computer Vision Library) は、コンピュータビジョン向けのオープンソースライブラリである\cite{OpenCV}。
本研究では、領域取得部(\ref{sec:area_coords_obtainment_part}節)における画像処理と領域の検出、文字情報取得部(\ref{sec:OCR_part}節)における画像処理に用いる。
具体的には、画像を読み込むimread関数、輪郭検出を行うfindContours関数やハフ変換を行うHoughLinesP関数などの関数を用いる。

\paragraph{imread関数}
imread関数は、画像ファイルを読み込む関数である。

\paragraph{cvtColor関数}
cvtColor関数は、画像の色空間を変換する関数である。
本研究では、画像をグレースケール化するために用いる。

\paragraph{GaussianBlur関数}
GaussianBlur関数は、画像内の白色ノイズを除去する関数である。
ガウシアンフィルタは、注目画素との距離に応じて重みを変えるガウシアンカーネルを用いた、画像の平滑化によって白色ノイズを除去するフィルタである。
標準偏差を0とすることで、カーネルの大きさから自動的に標準偏差を計算する\cite{ガウシアンフィルタ}。

\paragraph{threshold関数}
threshold関数は、画像を二値化する関数である。OpenCVでは、threshold関数の引数を指定することにより、二値化における閾値処理手法を選択することが可能である。
本提案手法では、THRESH\_TOZERO\_INVとTHRESH\_BINARYを、大津の二値化による閾値処理であるTHRESH\_OTSUを設定して二値化を行う。
大津の二値化(判別分析法)は、画素値のヒストグラムにおけるクラス間分散とクラス内分散の比である分離度が最大となる閾値を求める手法である\cite{大津の二値化}。
大津の二値化を用いることによって、閾値を自動で決定することが可能である。
THRESH\_TOZERO\_INVで大津の二値化を用いる場合は、大津の二値化によって決定した閾値を基準に、ある注目画素の画素値が閾値よりも小さい場合は変更せず、閾値よりも大きい場合は第三引数で設定する値に画素値を変換して二値化する。
THRESH\_BINARY\_INVで大津の二値化を用いる場合は、大津の二値化によって決定した閾値を基準に、ある注目画素の画素値が閾値よりも小さい場合は0に、閾値よりも大きい場合は第三引数で設定する値に画素値を変換して二値化する。
なお、一般に8ビット画像について、画素値は0に近づくほど黒を、255に近づくほど白を表すとされている\cite{画素値}。

\paragraph{findContours関数}

\paragraph{HoughLinesP関数}

\paragraph{Canny関数}
Canny関数は、Canny法によってエッジ検出を行う関数である。
エッジ検出は、画像内の輝度差を検出することにより、物体や領域の境界を識別する処理である\cite{エッジ検出}。
Canny法は、ガウシアンフィルタでノイズを除去し、ソーベルフィルタでエッジの勾配と方向を検出し、エッジを細線化した後に勾配と方向からエッジ検出を行う手法である。
Canny法におけるエッジ検出の閾値処理については、最低閾値と最大閾値を決め、画素値の微分値が最大閾値よりも大きい画素と、それらに隣接しており、かつ最低閾値よりも大きく最大閾値よりも小さい画素のみをエッジとみなすヒステリシス閾値処理を用いる。


\section{DeblurGANv2}\label{sec:Deblur-GANv2}
DeblurGANv2は、敵対的生成ネットワーク(GAN)をブレ除去に適用したツールである\cite{DeblurGANv2}。
スマートフォンで帳票画像を撮影する際に発生する画像内のブレを除去し、領域座標と文字認識の精度を高めることを目的として適用する。
また、スマートフォンのカメラで撮影した画像に対して複数の画像処理と共にDeblurGANv2を適用することで、ブレ除去によって画像品質が向上し、Tesseract-OCRを用いた文字認識の精度が向上することが先行研究で示されている\cite{DeblurGANv2の先行研究}。
本研究では、領域取得部(\ref{sec:area_coords_obtainment_part}節)と文字情報取得部(\ref{sec:OCR_part}節)で施す画像処理のひとつとして用いる。


\section{光学文字認識}\label{sec:Optical-Charactor-Recognition}


\section{Fugashi}\label{sec:Fugashi}
Fugashiは、形態素解析ソフトウェアであるMecabをPythonで使用する際のラッパーライブラリである\cite{Fugashi}。
本研究では、除外判定処理(\ref{subsec:exclusion_judgement_processing}節)で属性判定に不要な取得文字を出力の対象外とするために用いる。
また、本研究ではFugashiの解析に用いる辞書はUniDicであるため、本論文で記述する品詞体系についてはUniDic品詞体系とする。
UniDic品詞体系では、左からカンマ区切りで、大分類、中分類、小分類、細分類の順で品詞を列挙する\cite{UniDic品詞体系}。


\section{Youri}\label{sec:Youri}

\chapter{\toolName の機能}\label{cha:Function}



\section{記入欄の領域座標を取得する機能}\label{sec:area_coords_obtainment}


\section{記入欄にラベルを付与する機能}\label{sec:label_link}
\chapter{本ツールの実装}\label{cha:Implementation}
本ツールは、領域座標取得部、文字情報取得部、ラベル付与部、ファイル出力部の4つの処理部で構成する。
本ツールの構成を、図\ref{fig:structure}に示す。

以降、本章では4つの処理部について説明する。

\begin{figure}[t]
    \begin{center}
        \includegraphics[width=15cm]{image/04-implementation/structure.jpg}
        \caption{本ツールの構造}
        \label{fig:structure}
    \end{center}
\end{figure}


\section{領域座標取得部}\label{sec:area_coords_obtainment_part}
領域座標取得部は、帳票画像内にある記入欄を検出し、領域座標を取得、および、出力する。
矩形の帳票画像記入欄については、各頂点の4つのxy座標を、下線部の帳票画像記入欄については、両端点の2つのxy座標を領域座標として取得する。
領域座標取得部の出力結果である領域座標は、ラベル付与部(\ref{subsec:label_link_processing}節で後述)で用いる。

\subsection{矩形領域座標取得用画像処理}\label{subsec:image_processing_for_rect_coords_obtainment}
矩形領域座標取得用画像処理は、この後の処理である、矩形領域座標取得処理(\ref{subsec:rect_coords_obtainment_processing}節で後述)を実行するための前処理として、画像処理を行う。
本ツールでは、矩形の取得にあたり、OpenCVのfindContours関数(\ref{sec:OpenCV}節を参照)を用いる。
この関数を呼び出すとき、第一引数に渡す処理画像のパスについて、渡したパスの処理画像は白または黒色でなければエラーが発生してしまうため、処理画像を二値化する必要がある。
また、矩形の検出精度を高めるため、ノイズ除去(同節で後述)などの画像処理を組み合わせ、二値化する。

以下に、本処理で施す画像処理を順に示す。

\begin{enumerate}
    \item OpenCVのcvtColor関数を用いた、帳票画像のグレースケール化
        グレースケール化によって、色情報を削減し、計算量を減らすことで処理を高速化する。
    \item DeblurGANv2(\ref{sec:DeblurGANv2}節を参照)の適用によるブレ除去後のグレースケール化帳票画像の生成\\
        DeblurGANv2を適用することにより、帳票画像を撮影する際に発生する画像内のブレを除去し、矩形の検出精度を高める。
    \item OpenCVのGaussianBlur関数(\ref{sec:OpenCV}節を参照)を用いたガウシアンフィルタによるノイズ除去\\
        画像内のノイズを除去し、ノイズを矩形として検出することを防ぐ。
        なお、3行3列の矩形カーネルを使用し、標準偏差を0とする。
    \item OpenCVのthreshold関数(\ref{sec:OpenCV}節を参照)を用いた大津の二値化による二値画像への変換\\
        大津の二値化による閾値処理を行い、閾値の決定手法を指定する変数をTHRESH\_TOZERO\_INVに指定し、二値化する(\ref{sec:OpenCV}節を参照)。
        白黒を反転して二値化することにより、複数の矩形が隣接する場合に、それらを囲む矩形を不要に検出してしまうことを防ぐ。
    \item OpenCVのgetStructuringElement関数(\ref{sec:OpenCV}節を参照)を用いたカーネルの作成\\
        5行5列の矩形カーネルを作成する。
    \item OpenCVのdilate関数(\ref{sec:OpenCV}節を参照)を用いた膨張処理\\
        getStructuringElement関数で作成した矩形カーネルを用いて、。
        なお、膨張する回数は1回とする。
\end{enumerate}


\subsection{矩形領域座標取得処理}\label{subsec:rect_coords_obtainment_processing}
矩形領域座標取得処理は、矩形の帳票画像記入欄を検出し、矩形領域座標を取得、および、出力する処理である。
本処理の出力は、下線部領域座標取得処理(\ref{subsec:underline_coords_obtainment_processing}節で後述)の一部で利用する。

\ref{subsec:image_processing_for_rect_coords_obtainment}節で述べた画像処理後、OpenCVのfindContours関数(\ref{sec:OpenCV}節を参照)による輪郭検出を用いて矩形の帳票画像記入欄を検出する。
なお、以下の条件のいずれかに該当する矩形については、誤検知の可能性が高いとして、出力の対象外とする。

\begin{itemize}
    \item 面積が3000ピクセル以下である場合
    \item 一辺の長さが10ピクセル以下である場合
\end{itemize}



% スマートフォンのカメラで撮影する際にブレが生じた帳票画像を図\ref{fig:before_deblur}に示す。
% 図\ref{fig:before_deblur}に対して、DeblurGANv2を適用することによってブレを除去した画像を図\ref{fig:after_deblur}に示す。
% 図\ref{fig:before_deblur}と図\ref{fig:after_deblur}より、DeblurGANv2によるブレ除去によって、図\ref{fig:before_deblur}でブレている矩形の線が図\ref{fig:after_deblur}ではっきりとなっていることがわかる。

% \begin{figure}[t]
%     \centering
%     \begin{minipage}[t]{0.45\linewidth}
%       \centering
%       \fbox{
%         \includegraphics[keepaspectratio, width=7cm]{image/04-implementation/before_deblur.png}
%       }
%       \caption{撮影するにあたってブレが生じた帳票画像}
%       \label{fig:before_deblur}
%     \end{minipage}
%     \begin{minipage}[t]{0.45\linewidth}
%       \centering
%       \fbox{
%         \includegraphics[keepaspectratio, width=7cm]{image/04-implementation/after_deblur.jpeg}
%       }
%       \caption{発生したブレを除去した帳票画像}
%       \label{fig:after_deblur}
%     \end{minipage}
% \end{figure}


\subsection{下線部領域座標取得用画像処理}\label{subsec:image_processing_for_underline_coords_obtainment}
下線部領域座標取得用画像処理は、この後の処理である下線部領域座標取得処理(\ref{subsec:underline_coords_obtainment_processing}節で後述)を実行するための前処理として、画像処理を行う。
本ツールでは、下線部の取得にあたり、OpenCVのHoughLinesP関数(\ref{sec:OpenCV}節を参照)による、ハフ変換を用いた直線検出を行う。
この関数の第一引数に渡す処理画像のパスについて、渡すパスの処理画像は白または黒色でなければエラーが発生してしまうため、処理画像を二値化する必要がある。
また、直線の検出精度を高めるため、ノイズ除去などの画像処理を組み合わせ、二値化する。

以下に、本処理で施す画像処理を順に示す。
なお、本処理における画像処理の一部は、矩形領域座標取得処理(\ref{subsec:rect_coords_obtainment_processing}節を参照)と同様の画像処理を施す。

\begin{enumerate}
    \item OpenCVのcvtColor関数を用いた帳票画像のグレースケール化\\
        \ref{subsec:rect_coords_obtainment_processing}節で述べた処理と同様の処理
    \item DeblurGANv2の適用によるブレ除去後のグレースケール化帳票画像の生成\\
        \ref{subsec:rect_coords_obtainment_processing}節で述べた処理と同様の処理
    \item OpenCVのthreshold関数を用いた大津の二値化による二値画像への変換\\
        大津の二値化による閾値処理を行い、閾値の決定手法を指定する変数をTHRESH\_BINARY\_INVに指定し、二値化する。
        ハフ変換は白線を検出する処理であるため、白黒を反転して二値化することにより、黒色の直線を白色とし、検出精度を高める。
    \item OpenCVのCanny関数(\ref{sec:OpenCV}節を参照)を用いたCanny法によるエッジ検出\\
        閾値処理における上限と下限の閾値を、以下のように決定する。
        \begin{enumerate}
            \item 大津の二値化で取得した閾値を受け取る。
            \item 二値画像を対象に、画素値の中央値を定数倍(本研究では定数を0.33とする)し、取得した閾値から加減算する。
            \item 加算した値を上限の閾値、減算した値を下限の閾値として設定する。
        \end{enumerate}
\end{enumerate}


\subsection{下線部領域座標取得処理}\label{subsec:underline_coords_obtainment_processing}
下線部領域座標取得処理は、下線部の帳票画像記入欄を検出し、下線部領域座標を取得、および、出力する処理である。

\ref{subsec:image_processing_for_underline_coords_obtainment}節の画像処理後、OpenCVのHoughLinesP関数によるハフ変換を用いて、下線部の帳票画像記入欄を検出する。
なお、以下の条件のいずれかに該当する直線については、誤検出の可能性が高いとして、出力の対象外とする。
矩形領域座標取得処理(\ref{subsec:rect_coords_obtainment_processing}節を参照)の出力を、判定する条件の1つに用いる。

\begin{itemize}
    \item 直線の長さが10ピクセル未満である場合\\
        Canny法で適用するガウシアンフィルタで除去できていないノイズによって誤検出したエッジを下線部と捉えることを防ぐ。
    \item 水平を基準として傾きが3ピクセル以上である場合\\
        横書きの帳票において、下線部の直線は水平であるため、垂直な直線を下線部と捉えることを防ぐ。
    \item 直線が矩形領域の辺の一部から上下20ピクセル以内に存在する場合\\
        矩形領域座標取得処理の出力を判定に利用する。矩形領域の辺の一部を下線部と捉えることを防ぐ。
\end{itemize}

HoughLinesP関数によって直線を検出する際、入力画像内にある1本の直線の上下に、誤って2本の直線を検出してしまう不具合が発生する場合がある。
これは、両端点のxy座標が1ピクセル単位で異なる直線を、別の直線として検出するためである。
この不具合の発生を防ぐため、検出した直線の中点を全て計算し、ある直線における中点のy座標について、上下10ピクセル以内に別の直線の中点が存在する場合は、二直線の両端点のxy座標をそれぞれ平均して1本の直線に統一する。

\section{文字情報取得部}\label{sec:OCR_part}
文字情報取得部では、光学文字認識(\ref{sec:Optical-Charactor-Recognition}節を参照)によって、帳票画像内の文字情報を取得する。
本ツールでは、光学文字認識ソフトTesseract-OCR\cite{Tesseract-OCR}を用いる。
文字情報取得部の出力結果は、ラベル付与部(\ref{subsec:label_link_processing}節で後述)で用いる。

\subsection{文字情報取得用画像処理}\label{subsec:image_processing_for_char_recognition}
文字情報取得用画像処理は、この後の処理である文字情報取得処理(\ref{subsec:char_information_obtainment_processing}節で後述)を実行するための前処理として、文字の認識精度を高めるため、以下の順で帳票画像に画像処理を施す。

\begin{enumerate}
    \item DeblurGANv2の適用によるブレ除去後のグレースケール化帳票画像の生成\\
        \ref{subsec:rect_coords_obtainment_processing}節で述べた処理と同様の処理
    \item 領域座標取得部(\ref{sec:area_coords_obtainment_part}節を参照)から、領域座標を取得する。
    \item 背景色のRGBの値を取得\\
        背景色は、画像内で占める面積が最も広い色であるとする。
        全画素のRGB値を取得し、最も取得回数が多いRGB値を、背景色のRGB値とする。
    \item OpenCVのdrawContours関数(\ref{sec:OpenCV}節を参照)と、OpenCVのline関数(\ref{sec:OpenCV}節を参照)を用いた、背景色での矩形領域と下線部領域の描画\\
        矩形領域は矩形を、下線部領域は直線を、背景色で描画することによって、文字でない黒色の画素を減らす。
        これにより、文字でない黒色の画素値によって、大津の二値化で計算する閾値が変化することを防ぐ。
        本ツールでは、太さ15ピクセルの矩形と直線を描画する。
    \item OpenCVのimwrite関数を用いた、画像の保存\\
        背景色で矩形と直線を描画した画像を保存し、処理対象の画像を、入力である帳票画像から変更する。
    \item OpenCVのcvtColor関数を用いた、帳票画像のグレースケール化\\
        \ref{subsec:rect_coords_obtainment_processing}節で述べた処理と同様の処理
    \item OpenCVのthreshold関数を用いた、大津の二値化による二値画像への変換\\
        大津の二値化による閾値処理を行い、閾値の決定手法を指定する変数をTHRESH\_BINARYに指定し、二値化する。
\end{enumerate}

\subsection{文字情報取得処理}\label{subsec:char_information_obtainment_processing}
文字情報取得処理は、文字情報を取得する処理である。
\ref{subsec:image_processing_for_char_recognition}節の画像処理後、Tesseract-OCRによる文字認識を行う。
本ツールでは、PythonのOCR用のラッパーライブラリであるPyOCR\cite{PyOCR}から、変数builderにLineBoxBuilderを指定し、行単位で文字認識を行う。
これによって、取得文字と文字位置を行単位で取得することができる。

文字情報取得後、バウンディングボックスの左上頂点のy座標を参照し、昇順にソートし、番号を0から順に割り振る。
y座標が同じ場合は、さらにx座標を参照し、昇順にソートする。
本来は、y座標が同じ場合は、x座標を昇順にソートするため、左から右へ番号が大きくなる。
しかし、人間の目視で複数の文字が同じ行に存在すると認識するとき、割り振った番号を参照すると、昇順にならない不具合が発生する場合がある。
ラベル付与部(\ref{subsec:label_link_processing}節で後述)で文字位置を扱う際に、この不具合が起こった場合、ラベルの更新順が変化することで、異なるラベルを領域座標に割り付ける可能性がある。

ある帳票画像に対して文字位置取得処理を施し、同行に存在する文字列に対して、割り振った番号が昇順にならない場合の画像を、図\ref{fig:before_sorted_string}に示す。
図\ref{fig:before_sorted_string}内で描画している赤い矩形は、取得文字を囲むバウンディングボックスであり、バウンディングボックスの左上に、ソート後に割り振った番号を表示している。
図\ref{fig:before_sorted_string}の上部の文字に対して割り振った番号は、左から14番、15番、13番、12番となっている。
本来は、割り振った番号が左から12番、13番、14番、15番となる必要がある。

\begin{figure}[t]
    \begin{center}
        \fbox{
            \includegraphics[width=15cm]{image/04-implementation/before_sorted_string.png}
        }
        \caption{誤って番号を割り振った文字位置}
        \label{fig:before_sorted_string}
    \end{center}
\end{figure}

これは、Tesseract-OCRが文字を認識する順番を、y座標についてピクセル単位で昇順にソートするため、人間の目視で認識する順番とソート後の順番に違いが生じるためである。
以下に、この不具合の発生を防ぐため、再ソートを行う流れを示す。
以下の処理は、iを0に初期化し、文字を認識した順番で、取得文字の数だけ繰り返す。

\begin{enumerate}
    \item i番目のバウンディングボックスの左上頂点のy座標を取得する。
    \item 取得したy座標を基準として、10ピクセル以内に別のバウンディングボックスの左上頂点が存在する場合は、以下の処理を順に行う。
    \begin{enumerate}
        \item 条件にあてはまる左上頂点のxy座標全てを、空のリストgroupに格納する。
        \item リストgroup内について、x座標の値を参照して昇順にソートする。
        \item iをリストgroupの要素数だけ増やす。
        \item リストgroupの要素を全て削除する。
    \end{enumerate}
\end{enumerate}

これによって、人間の目視で同じ行に存在すると認識する複数の文字を対象に、文字位置取得時点のy座標について、最小のy座標と最大のy座標の差が10ピクセル以内であれば、正しくソートができる。
再ソート後にバウンディングボックスを描画した画像を、図\ref{fig:after_sorted_string}に示す。
ソート後は、図\ref{fig:before_sorted_string}では不具合が発生していた箇所が、図\ref{fig:after_sorted_string}では、左から12番、13番、14番、15番となっており、昇順となっていることがわかる。

\begin{figure}[t]
    \begin{center}
        \fbox{
            \includegraphics[width=15cm]{image/04-implementation/after_sorted_string.png}
        }
        \caption{再ソートによって昇順に並び替えた文字位置}
        \label{fig:after_sorted_string}
    \end{center}
\end{figure}

図\ref{fig:original}に示した帳票画像に対する、文字情報取得処理の出力をまとめたものを、図\ref{fig:char_recognition_for_original}に示す。
また、図\ref{fig:char_recognition_for_original}で示す文字位置を参照し、バウンディングボックスを描画した画像を、図\ref{fig:bbox_recognition_for_original}に示す。
図\ref{fig:char_recognition_for_original}で表示している内容を、図に示す。

% \begin{center}
%     \

% \begin{enumerate}
%     \item \ref{subsec:char_information_obtainment_processing}節で割り振った、ソート後の番号
%     \item バウンディングボックスの左上頂点のxy座標
%     \item バウンディングボックスの右下頂点のxy座標
%     \item 取得文字
% \end{enumerate}

例えば、番号1の取得文字は、「請求日」であり、バウンディングボックスの左上頂点のxy座標が、(1597, 321)であり、バウンディングボックスの右下頂点のxy座標が、(1724, 363)となる。
図\ref{fig:char_recognition_for_original}の番号は、図\ref{fig:bbox_recognition_for_original}のバウンディングボックスの左上に表示する番号と一致する。

\lstset{language=}
\begin{figure}[t]
    \begin{lstlisting}
    string[0] ((1644, 244), (1730, 285)) : 番号
    string[1] ((1597, 321), (1724, 363)) : 請求日
    string[2] ((1343, 486), (1411, 528)) : 詩
    string[3] ((1046, 518), (1106, 555)) : 月
    string[4] ((1194, 487), (1260, 555)) : 求
    string[5] ((1354, 530), (1400, 555)) : 較
    string[6] ((898, 683), (983, 725)) : 御中
    string[7] ((223, 992), (926, 1034)) : 下記の通り、ご請求申し上げます。
    string[8] ((1573, 1073), (1662, 1106)) : FAX
    string[9] ((229, 1147), (449, 1189)) : ご請求金額
    string[10] ((1126, 1147), (1248, 1190)) : (税込)
    string[11] ((1574, 1146), (1656, 1187)) : 担当
    string[12] ((1183, 1313), (1270, 1354)) : 数量
    string[13] ((1526, 1311), (1606, 1352)) : 単価
    string[14] ((1970, 1311), (2058, 1352)) : 合計
    string[15] ((1521, 2267), (1608, 2308)) : 小計
    string[16] ((1499, 2343), (1632, 2386)) : 消費税
    string[17] ((1520, 2422), (1608, 2463)) : 合計
    string[18] ((1160, 2576), (1293, 2619)) : 備考
    \end{lstlisting}
    \caption{図\ref{fig:original}から取得した文字}
    \label{fig:char_recognition_for_original}
\end{figure}

バウンディングボックスを描画していない文字については、認識できなかった文字である。
認識に失敗した文字が及ぼす影響については、\ref{sec:problems}節で後述する。

\begin{figure}[t]
    \begin{center}
        \fbox{
            \includegraphics[width=15cm]{image/04-implementation/char_with_bbox.png}
        }
        \caption{図\ref{fig:char_recognition_for_original}の文字位置を図\ref{fig:original}の画像に描画した画像}
        \label{fig:bbox_recognition_for_original}
    \end{center}
\end{figure}

\subsection{除外判定処理}\label{subsec:exclusion_judgement_processing}
除外判定処理は、Fugashi(\ref{sec:Fugashi}節を参照)による形態素解析を行い、属性推測処理(\ref{subsec:att_prediction_processing}節で後述)に不要な取得文字について、出力から除外する処理である。
形態素解析ソフトウェアであるMecab(\ref{sec:Fugashi}節を参照)を用いて、取得文字を形態素に分割し、品詞を解析する。
取得文字を構成する形態素の数のうち、特定の品詞である形態素数の割合が半分以上である場合は、属性判定において意味がない取得文字であるとして、該当の取得文字と文字位置を、文字情報から除外する。
文字を認識する際に、紙面と背景の境界や、領域取得部(\ref{sec:area_coords_obtainment_part}節を参照)で取得できなかった矩形や直線を、文字として誤認識する場合がある。
不要な文字の属性推測処理を防ぐことにより、処理時間を短縮することができる。

以下に、UniDic品詞体系(左からカンマ区切りで、大分類、中分類、小分類、細分類)をもとに、除外対象である形態素の品詞を示す。
なお、除外対象とする品詞は、経験から決定している。

\begin{itemize}
    \item 補助記号,一般,*,*
    \item 感動詞,一般,*,*
    \item 感動詞,フィラー,*,*
\end{itemize}

ある帳票画像に対して、\ref{subsec:char_information_obtainment_processing}節で取得したバウンディングボックスを描画した画像の一部を、図\ref{fig:before_exclusion_bbox}に示す。
また、図\ref{fig:before_exclusion_bbox}で描画したバウンディングボックスに対応する取得文字を出力した画像を、図\ref{fig:before_exclusion_string}に示す。
図\ref{fig:before_exclusion_bbox}と図\ref{fig:before_exclusion_string}の58番および60番の取得文字は、帳票画像内の矩形の辺を誤って文字として認識している。
この矩形については、文字情報取得用画像処理(\ref{subsec:image_processing_for_char_recognition}節を参照)で取得できなかったため、背景色の矩形を描画できていない。
図\ref{fig:before_exclusion_bbox}と図\ref{fig:before_exclusion_string}の58番および60番の取得文字は、属性推測処理において意味がない文字である。
図\ref{fig:before_exclusion_string}に示した取得文字に対して除外判定処理を適用し、属性推測処理に不要な取得文字を除外した出力を、図\ref{fig:after_exclusion_string}に示す。
図\ref{fig:after_exclusion_string}より、属性推測処理に不要な取得文字の除外に成功していることがわかる。

\begin{figure}[t]
    \begin{center}
        \fbox{
            \includegraphics[width=15cm]{image/04-implementation/before_exclusion_bbox.png}
        }
        \caption{属性判定に不要な文字を含む文字認識}
        \label{fig:before_exclusion_bbox}
    \end{center}
\end{figure}

\begin{figure}[t]
    \begin{center}
        \includegraphics[width=15cm]{image/04-implementation/before_exclusion_string.png}
        \caption{除外判定処理適用前の取得文字}
        \label{fig:before_exclusion_string}
    \end{center}
\end{figure}

\begin{figure}[t]
    \begin{center}
        \includegraphics[width=15cm]{image/04-implementation/after_exclusion_string.png}
        \caption{除外判定処理適用後の取得文字}
        \label{fig:after_exclusion_string}
    \end{center}
\end{figure}

\section{ラベル付与部}\label{sec:label_link_part}
ラベル付与部では、除外判定処理(\ref{subsec:exclusion_judgement_processing}節を参照)後の取得文字の属性を、属性の候補(\ref{subsec:att_prediction}節を参照)である、日付(date)、文字列(string)、数値(number)の3つから適切な1つを推測し、取得した領域にラベルとして付与する。
付与するラベルの種類は、領域近傍の取得文字から推測する属性に依存する。
領域座標と、領域座標に対応するラベルを組としたJSONファイルを出力とする。

\subsection{属性推測処理}\label{subsec:att_prediction_processing}
属性推測処理では、取得文字に対して、属性の候補である、日付(date)、文字列(string)、数値(number)の中から推測する。
なお、属性が推測不可である取得文字の属性は、文字列とする。
属性の推測には、大規模言語モデルYouri(\ref{sec:Youri}節を参照)を用いる。
YouriはLlama2を日本語の学習データで継続事前学習を行った大規模言語モデルである。
大規模言語モデルは、指示と異なる出力をする場合がある。
候補ではない属性の推測を防ぐため、Youriの出力から3つの属性のいずれかとなるよう補正を行う。
日本語の推論に特化した言語モデルを利用することによって、取得した日本語の文字に対して、属性をより正確に推測することができる。

以下に、属性を推測する処理の流れを示す。
以下の処理は、文字情報取得処理(\ref{subsec:char_information_obtainment_processing}節を参照)でソートを行った順番で、除外判定処理(\ref{subsec:exclusion_judgement_processing}節を参照)後の取得文字の数だけ繰り返す。

\begin{enumerate}
    \item 除外判定処理(\ref{subsec:exclusion_judgement_processing}節を参照)の出力である除外判定後の取得文字を受け取る。
    \item 以下のプロンプトを入力として属性を推測する。なお、(取得文字)は、除外判定後の取得文字を指す。\\
    \begin{quotation}
        \# 命令書:\\
        以下の制約条件にあてはまるものを出力せよ。
        
        \# 制約条件:\\
        ・記入欄に記入する内容が、日付、文字列、数値の中から、どのデータ型が最も適切であるかを選択する。\\
        ・出力は短く、あてはまるデータ型のみとする。\\
        ・例として、年月日などは日付、氏名などは文字列、金額などは数値があてはまる。\\
        (取得文字)という欄は、どのデータ型に該当するか。
    \end{quotation}
    \item 以下の順で処理を行い、属性を補正する。
        \begin{enumerate}
            \item 全文字の属性を文字列(string)とする。
            \item 出力に「日」を含む場合は、日付(date)として判定し、属性を文字列(string)から更新し、属性を補正する。
            \item 出力に「数」を含む場合は、数値(number)として判定し、属性を文字列(string)から更新し、属性を補正する。
            \item 出力に「日」、「数」を含まない場合は、属性を更新せず、文字列(string)とする。
        \end{enumerate}
\end{enumerate}

図\ref{fig:char_recognition_for_original}に示した取得文字に対して、Youriが出力したテキストの一部を、以下の図\ref{fig:output_Youri}に示す。
図\ref{fig:output_Youri}は、左にYouriの出力を、右の括弧内に属性を推測した取得文字を示す。
図\ref{fig:output_Youri}の1行目や4行目は、「番号」、「月」という取得文字に対して、属性の候補ではなく、文を出力している。

\lstset{language=}
\begin{figure}[t]
    \begin{lstlisting}
        数値に該当する。 (番号)
        日付 (請求日)
        文字列 (詩)
        文字列が適切です。 (月)
        日付 (求)
        数値 (較)
        日付 (御中)
        日付 (下記の通り、ご請求申し上げます。)
        数字です。 (FAX)
        数値 (ご請求金額)
        「(税込)」は、数値のデータ型に該当します。 ((税込))
        文字列 (担当)
        数値 (数量)
        数値型に該当する (単価)
        数値 (合計)
        数値 (小計)
        数値 (消費税)
        数値 (合計)
        備考は、文字列に該当する。 (備考)
    \end{lstlisting}
    \caption{図\ref{fig:char_recognition_for_original}に示した取得文字からYouriが出力したテキスト}
    \label{fig:output_Youri}
\end{figure}

図\ref{fig:output_Youri}で示したYouriの出力から、属性を補正した結果を、図\ref{fig:predict_att_for_original}に示す。
図\ref{fig:predict_att_for_original}は、左から、\ref{subsec:char_information_obtainment_processing}節で割り振った、ソート後の番号、属性を判定した結果、推測対象である取得文字の順で表す。
例えば、番号1の取得文字は、「請求日」であり、推測した属性は日付(date)となる。番号4の取得文字は、「月」であり、推測した属性は文字列(string)となる。
図\ref{fig:predict_att_for_original}の番号は、図\ref{fig:bbox_recognition_for_original}のバウンディングボックスの左上頂点に表示する番号、および図\ref{fig:char_recognition_for_original}で示した番号と一致する。
なお、大規模言語モデルによる出力に依存して属性を決定するため、常に正しい属性を判定することはできていない。
例えば、番号6の取得文字である「御中」の属性を、日付(date)と判定している。
御中は組織や団体の敬称であり、本来は文字列(string)の属性が正しい。

\lstset{language=}
\begin{figure}[t]
    \begin{lstlisting}
        att[0]: number (番号)
        att[1]: date (請求日)
        att[2]: string (詩)
        att[3]: date (月)
        att[4]: date (求)
        att[5]: number (較)
        att[6]: date (御中)
        att[7]: date (下記の通り、ご請求申し上げます。)
        att[8]: number (FAX)
        att[9]: number (ご請求金額)
        att[10]: number ((税込))
        att[11]: string (担当)
        att[12]: number (数量)
        att[13]: number (単価)
        att[14]: number (合計)
        att[15]: number (小計)
        att[16]: number (消費税)
        att[17]: number (合計)
        att[18]: string (備考)
    \end{lstlisting}
    \caption{図\ref{fig:original}に対して取得した文字情報}
    \label{fig:predict_att_for_original}
\end{figure}

\subsection{ラベル割付処理}\label{subsec:label_link_processing}
ラベル割付処理では、領域座標取得部(\ref{sec:area_coords_obtainment_part}節を参照)で取得した領域座標に対して、近傍に存在する文字の属性を割り付ける。
属性推測処理(\ref{subsec:att_prediction_processing}節を参照)で推測した属性と、文字情報取得処理(\ref{subsec:char_information_obtainment_processing}節を参照)で取得した文字位置をもとに、取得文字近傍の領域座標に対して、推測した属性をラベルとして割り付ける。

以下に、領域座標にラベルを割り付ける流れを示す。
以下の処理は、文字位置取得処理でソートを行った順番で、除外判定処理(\ref{subsec:exclusion_judgement_processing}節を参照)後の取得文字の数だけ繰り返す。

\begin{enumerate}
    \item \label{enum:bbox_center} 文字位置であるバウンディングボックスの中心点のxy座標を計算する。
    \item 取得した領域座標のうち、矩形領域は右下頂点のxy座標、下線部領域は右端点のxy座標と、計算した中心点のxy座標を比較し、\ref{enum:bbox_center}で計算した中心点のx座標とy座標が共に大きい全ての領域座標をラベル割付の対象として、文字位置に対応する取得文字の属性をラベルとして割り付ける。
    \item 繰り返し処理によって、既にラベルを割り付けた領域座標がラベル割付の対象となった場合は、後の処理で割り付けるラベルに更新する。
\end{enumerate}

以上の繰り返し処理を行った後、領域座標取得部(\ref{sec:area_coords_obtainment_part}節を参照)で取得した領域座標に対して、ラベルを割り付ける。

\section{ファイル出力部}\label{subsec:file_output_part}
ファイル出力部では、領域座標と、対応するラベルを組とするJSON形式のファイルと、取得領域を強調表示したPNG形式の画像2枚を出力する。
JSONファイルと、PNG画像のエクスポートにあたって、領域座標取得部(\ref{sec:area_coords_obtainment_part}節を参照)で取得した領域座標と、ラベル割付処理(\ref{subsec:label_link_processing}節を参照)で取得したラベルを参照する。


\subsection{JSONファイル出力処理}\label{subsec:json_file_output_processing}
JSONファイル出力処理では、取得した領域座標とラベルを整形し、領域座標と、対応するラベルを組とするJSONファイルを出力する処理である。
出力するJSONファイルは、配列rects\_dataと、配列underlines\_dataで構成し、矩形領域の情報、下線部領域の情報をそれぞれ持つ。
これら2つの配列は、それぞれ以下の情報をもつ。

\begin{itemize}
    \item 配列ごとに一意であるid
    \item 領域に割り付けているラベルを示すlabel
    \item 領域座標を示すオブジェクトcoords
\end{itemize}

配列rects\_dataと、配列underlines\_dataを、図\ref{fig:example_output_json}に示す形式に整形後、JSONファイルを出力する。

\subsection{領域強調画像出力処理}\label{subsec:area_highlighted_image_output_processing}
領域強調画像出力処理では、領域座標取得部(\ref{sec:area_coords_obtainment_part}節を参照)で取得した領域座標と、ラベル割付処理(\ref{subsec:label_link_processing}節を参照)で取得したラベルを参照し、矩形と直線を、割り付けたラベルと共に描画した画像を出力する。
出力する画像については、矩形領域強調画像と、下線部領域強調画像の計2枚の画像を出力する。

矩形領域強調画像については、矩形をランダムなRGBカラーで描画することで、矩形領域を強調表示する。
さらに、左上頂点に、JSONファイル内のrect\_data配列のidキーに対応する値と、labelキーに対応する値を表示する。
下線部の帳票画像記入欄については、緑色で下線部領域の直線を描画することによって、下線部領域を強調表示する。
さらに、左端点の上に、JSONファイル内のunderlines\_data配列のidキーに対応する値と、labelキーに対応する値を表示する。
これにより、人間が目視でJSONファイルを確認する場合と比較して、取得した領域座標と割り付けたラベルの確認が容易となる。

以下に、矩形と直線を、割り付けたラベルと共に描画した画像を出力する流れを示す。

\begin{enumerate}
    \item \ref{sec:area_coords_obtainment_part}節と\ref{subsec:label_link_processing}節から、領域座標と、領域座標に対応するラベルを、それぞれ取得する。
    \item Pythonのcopy関数により、入力である帳票画像をコピーし、2枚の帳票画像A、Bを生成する。
    \item 帳票画像Aに対して、OpenCVのdrawContours関数を用いて、矩形領域座標を参照して矩形を描画する。
    \item 帳票画像Bに対して、OpenCVのline関数を用いて、下線部領域座標を参照して直線を描画する。
    \item OpenCVのputText関数(\ref{sec:OpenCV}節を参照)を用いて、idキーに対応する値と、labelキーに対応する値を、各領域座標の左上に描画する。
    \item OpenCVのimwrite関数を用いて、帳票画像Aと帳票画像Bを、それぞれ矩形領域強調画像と下線部領域強調画像として保存する。
\end{enumerate}

以上の処理後、矩形領域強調画像と、下線部領域強調画像を出力する。
\chapter{適用例}\label{cha:Indication}
本章では、本研究で試作したツールが正しく動作することを確認する。
適用例として、試作したツールを適用する帳票画像を、図\ref{fig:indication_original}に示す。

\begin{figure}[tp]
    \begin{center}
        \fbox{
            \includegraphics[width=15cm]{image/05-indication/indication_original.jpg}
        }
        \caption{試作したツールを適用する帳票画像}
        \label{fig:indication_original}
    \end{center}
\end{figure}

図\ref{fig:indication_original}に対して、試作したツールを適用し、出力であるJSONファイルと、矩形領域強調画像と下線部領域強調画像の2枚が、実際の矩形領域と下線部領域の位置とラベルがそれぞれ一致することを確認する。
具体的には、JSONファイルのrects\_data配列と、矩形領域強調画像を参照し、矩形領域の出力結果を確認する。
同様に、JSONファイル内のunderlines\_data配列と、下線部領域強調画像を参照し、下線部領域の出力結果を確認する。

\section{矩形領域についての出力結果}\label{sec:result_rect}
本節は、矩形領域についての出力結果を確認する。

図\ref{fig:indication_original}に対して、試作したツールを適用し、出力したJSONファイルのうち、rects\_data配列の一部を、図\ref{fig:rects_data_json}に示す。

\lstset{language=}
\begin{figure}[tp]
    \begin{lstlisting}
        {
            "id": 4,
            "label": "string",
            "coords": {
                "top_left": {
                    "x": 275,
                    "y": 817
                },
                "buttom_left": {
                    "x": 275,
                    "y": 903
                },
                "buttom_right": {
                    "x": 1008,
                    "y": 903
                },
                "top_right": {
                    "x": 1008,
                    "y": 817
                }
            }
        },
        {
            "id": 5,
            "label": "number",
            "coords": {
                "top_left": {
                    "x": 1016,
                    "y": 817
                },
                "buttom_left": {
                    "x": 1016,
                    "y": 903
                },
                "buttom_right": {
                    "x": 1308,
                    "y": 903
                },
                "top_right": {
                    "x": 1308,
                    "y": 817
                }
            }
        },
    \end{lstlisting}
    \caption{rects\_data配列の一部}\label{fig:rects_data_json}
\end{figure}

図\ref{fig:rects_data_json}のJSONファイルと同時に出力した2枚の領域強調画像のうち、矩形領域を強調した画像の一部を、図\ref{fig:highlighted_rects_part}に示す。

\begin{figure}[tp]
    \begin{center}
        \fbox{
            \includegraphics[width=15cm]{image/05-indication/highlighted_rects_part.jpg}
        }
        \caption{矩形領域を強調した画像の一部}
        \label{fig:highlighted_rects_part}
    \end{center}
\end{figure}

矩形領域座標について、IoU(Intersection over Union)を用いてJSONファイルの出力が正しいことを確認する。

実際の矩形領域座標と、出力するJSONファイルの矩形領域座標を確認し、IoUを算出する。
図\ref{fig:rects_data_json}より、idキーの値が4である矩形領域座標は、左上頂点のxy座標が(275,817)であり、右下頂点のxy座標が(1008,903)である。
図\ref{fig:highlighted_rects_part}にある4番の矩形領域について、図\ref{fig:indication_original}の画像では、左上頂点のxy座標が(273,816)であり、右下頂点のxy座標が(1009,905)であった。
よって、IoUを算出すると、0.96となる。
IoUが閾値である0.5以上であるため、この矩形領域座標は正しく取得できたと言える。
他の矩形領域に対しても、矩形領域座標を正しく取得していることを確認した。

ラベルについて、人間が判断したラベルと、ツールが出力したラベルが一致するかを確認する。
図\ref{fig:rects_data_json}より、idキーの値が4と5である矩形領域について、それぞれラベルはstring、numberである。
図\ref{fig:highlighted_rects_part}より、画像内に描画した矩形領域の番号のうち、4番の矩形領域は「品名」を記入する欄であり、5番の矩形領域は「数量」を記入する欄であることがわかる。
各矩形領域のラベルについて、「品名」は文字列(string)、「数量」は数値(number)がそれぞれ正しいラベルであるため、これら2つの矩形領域については、正しいラベルを割り付けていることを確認できた。
他の矩形領域に対しても、一部を除き、正しいラベルを割り付けていることを確認した。

また、矩形領域について、JSONファイルと領域強調画像の出力が対応することから、帳票画像のレイアウトを変更せず、矩形領域の領域座標とラベルを出力できたことがわかる。

一部の矩形領域については、本来割り付けるべきラベルとは異なるラベルを誤って割り付けた。
JSONファイルのうち、誤ったラベルを割り付けた矩形領域座標を、図\ref{fig:rects_data_miss_json}に示す。

\lstset{language=}
\begin{figure}[tp]
    \begin{lstlisting}
        {
            "id": 25,
            "label": "string",
            "coords": {
                "top_left": {
                    "x": 1713,
                    "y": 1283
                },
                "buttom_left": {
                    "x": 1713,
                    "y": 1369
                },
                "buttom_right": {
                    "x": 2262,
                    "y": 1369
                },
                "top_right": {
                    "x": 2262,
                    "y": 1283
                }
            }
        },
    \end{lstlisting}
    \caption{誤ったラベルを割り付けた矩形領域座標}\label{fig:rects_data_miss_json}
\end{figure}

また、図\ref{fig:rects_data_miss_json}に示した矩形領域座標に対応する矩形領域強調画像を、図\ref{fig:highlighted_rects_miss_part}に示す。

\begin{figure}[tp]
    \begin{center}
        \fbox{
            \includegraphics[width=15cm]{image/05-indication/highlighted_rects_miss_part.jpg}
        }
        \caption{図\ref{fig:rects_data_miss_json}の矩形領域を描画した矩形領域強調画像}
        \label{fig:highlighted_rects_miss_part}
    \end{center}
\end{figure}

図\ref{fig:rects_data_miss_json}より、idキーの値が25の矩形領域は、labelキーの値がstringであることがわかる。
図\ref{fig:highlighted_rects_miss_part}より、25番の矩形領域は、「小計」を記入する欄であることがわかる。
本来割り付けるべきラベルは数値(number)であるため、誤ったラベルを割り付けていることがわかる。
これは、「小計」という文字を認識することができなかったことが原因である。

図\ref{fig:indication_original}の画像に対して、認識した文字のバウンディングボックスを描画した画像の一部を、図\ref{fig:OCR_result}に示す。

\begin{figure}[tp]
    \begin{center}
        \fbox{
            \includegraphics[width=15cm]{image/05-indication/OCR_result.png}
        }
        \caption{図\ref{fig:indication_original}で認識した文字}
        \label{fig:OCR_result}
    \end{center}
\end{figure}

図\ref{fig:OCR_result}の右にある「小計」については、バウンディングボックスを描画していないため、光学文字認識に失敗していることがわかる。
また、図\ref{fig:OCR_result}の左にある「備考」の文字について、属性を文字列(string)と推測したことを確認した。
このことから、光学文字認識を失敗したことによって、本来対応する文字とは別の文字の属性をラベルとして割り付けたことが、誤ったラベルを割り付けた原因であることがわかる。
この問題点については、\ref{sec:problems}節で後述する。

\section{下線部領域についての出力結果}\label{sec:result_underline}
本節は、下線部領域についての出力結果を確認する。

図\ref{fig:indication_original}に対して、試作したツールを適用し、出力したJSONファイルのうち、underlines\_data配列の一部を、図\ref{fig:underlines_data_json}に示す。
また、図\ref{fig:underlines_data_json}のJSONファイルと同時に出力した2枚の下線部強調画像のうち、下線部領域を強調した画像の一部を図\ref{fig:highlighted_underlines_part}に示す。

\lstset{language=}
\begin{figure}[tp]
    \begin{lstlisting}
        {
            "id": 0,
            "label": "date",
            "left": {
                "x": 869,
                "y": 354
            },
            "right": {
                "x": 1512,
                "y": 354
            }
        },
        {
            "id": 1,
            "label": "number",
            "left": {
                "x": 1908,
                "y": 355
            },
            "right": {
                "x": 2265,
                "y": 355
            }
        },
    \end{lstlisting}
    \caption{underlines\_data配列の一部}\label{fig:underlines_data_json}
\end{figure}

\begin{figure}[tp]
    \begin{center}
        \fbox{
            \includegraphics[width=15cm]{image/05-indication/highlighted_underlines_part.jpg}
        }
        \caption{下線部領域を強調した画像の一部}
        \label{fig:highlighted_underlines_part}
    \end{center}
\end{figure}

下線部領域座標について、矩形領域座標と同様に、IoUを用いてJSONファイルの出力が正しいことを確認する。
なお、下線部領域については、高さを3ピクセルとしてIoUを計算する。
実際の下線部領域座標と、出力するJSONファイルの下線部領域座標を確認し、IoUを算出する。
図\ref{fig:underlines_data_json}より、idキーの値が0である下線部領域座標は、左端点のxy座標が(869,354)であり、右端点のxy座標が(1512,354)である。
図\ref{fig:highlighted_underlines_part}にある0番の下線部領域について、図\ref{fig:indication_original}の画像では、左端点のxy座標が(869,354)であり、右端点のxy座標が(1511,354)であった。
よって、IoUを算出すると、0.99となる。
IoUが閾値である0.5以上であるため、この下線部領域座標は正しく取得できたと言える。
他の下線部領域に対しても、下線部領域座標を正しく取得していることを確認した。

下線部領域のラベルについて、人間が判断したラベルと、ツールが出力したラベルが一致するかを確認する。
図\ref{fig:underlines_data_json}より、idキーの値は0と1であり、それぞれラベルはdate、numberである。
図\ref{fig:highlighted_underlines_part}より、画像内に描画した矩形領域の番号のうち、0番の下線部領域は「年月日」を記入する欄であり、1番の下線部領域は番号である「No.」を記入する欄であることがわかる。
「年月日」は日付(date)、「No.」は数値(number)がそれぞれ正しいラベルであるため、これら2つの下線部領域については、正しいラベルを割り付けていることを確認できた。
他の下線部領域に対しても、一部を除き、正しいラベルを割り付けていることを確認した。

また、下線部領域について、JSONファイルと領域強調画像の出力が対応することから、帳票画像のレイアウトを変更することなく、下線部領域の領域座標とラベルを出力できたことがわかる。

一部の下線部領域については、本来割り付けるべきラベルとは異なるラベルを誤って割り付けた。
JSONファイルのうち、誤ったラベルを割り付けた下線部領域座標を、図\ref{fig:underlines_data_miss_json}に示す。

\lstset{language=}
\begin{figure}[tp]
    \begin{lstlisting}
        {
            "id": 2,
            "label": "date",
            "left": {
                "x": 273,
                "y": 615
            },
            "right": {
                "x": 1312,
                "y": 615
            }
        },
    \end{lstlisting}
    \caption{誤ってラベルを割り付けた下線部領域}\label{fig:underlines_data_miss_json}
\end{figure}

図\ref{fig:underlines_data_miss_json}の下線部領域は、図\ref{fig:highlighted_underlines_part}の2番の下線部領域である。
図\ref{fig:underlines_data_miss_json}より、idキーの値が2の下線部領域は、labelキーの値がdateであることがわかる。
図\ref{fig:highlighted_underlines_part}の左下を確認すると、2番の下線部領域は、右にある「様」という文字から、宛名を記入する欄であることがわかる。
本来割り付けるべきラベルは文字列(string)であるため、誤ったラベルを割り付けていることがわかる。
これは、\ref{subsec:label_link_processing}節で述べたラベルを割り付ける手法では、領域座標の右にある文字の属性を参照しないことが原因である。
この問題点については、\ref{sec:problems}節で後述する。
\chapter{考察}\label{cha:Discussion}
本研究では、レイアウトを変更せず、電子フォーム作成にかかる手間と時間を削減することを目的として、記入欄自動検出およびラベル割付ツールの試作を行った。
試作したツールは、以下に示す2つの機能を持つ。

\begin{itemize}
  \item 領域座標自動取得およびラベル割付機能
  \item 領域強調画像出力機能
\end{itemize}

本章では、評価実験を行い、試作したツールの有用性を評価する。
また、試作したツールと関連ツールを比較する。
最後に、試作したツールの問題点について述べる。

\section{試作したツールの有用性に関する評価}\label{sec:evalue_usefulness}
試作したツールの有用性を評価するため、試作したツールを、電子フォーム作成ツールであるPhotolize\cite{Photolize}に適用し、実験を行う。
Photolizeは、スマートフォンで撮影した帳票の画像に対して、電子フォーム内記入欄をGUIツールで配置することによって、電子フォームの作成を支援するWebアプリケーションである。
試作したツールをPhotolizeに適用することで、領域座標とラベルをまとめたJSONファイルを参照し、電子フォーム内記入欄を自動で配置する。
これによって、電子フォーム内記入欄の配置にかかる手間と時間を削減することができる。

実験の対象とした帳票の画像を、図\ref{fig:experiment_A}と図\ref{fig:experiment_B}に示す。
\begin{figure}[tp]
    \begin{center}
        \fbox{
            \includegraphics[width=15cm]{image/06-discussion/experiment_A.jpg}
            }
        \caption{実験対象である電子文書の帳票画像A}
        \label{fig:experiment_A}
    \end{center}
\end{figure}

\begin{figure}[tp]
    \begin{center}
        \fbox{
            \includegraphics[width=15cm]{image/06-discussion/experiment_B.jpg}
        }
        \caption{実験対象である電子化文書の帳票画像B}
        \label{fig:experiment_B}
    \end{center}
\end{figure}
図\ref{fig:experiment_A}は、電子文書の画像であり、図\ref{fig:experiment_B}は、電子化文書の画像である。
以降、図\ref{fig:experiment_A}の画像を、帳票画像A、図\ref{fig:experiment_B}の画像を、帳票画像Bと呼ぶ。
なお、帳票画像内記入欄の数は、それぞれ図\ref{fig:experiment_A}が54個、図\ref{fig:experiment_B}が48個である。

実験は、宮崎大学工学部情報システム工学科に所属する学部4年生2名、および修士1年生3名、2年生1名の計6名を対象として行った。

実験方法は、被験者6名をグループ$\alpha$とグループ$\beta$の2グループに分けて行う。
グループ$\alpha$は、図\ref{fig:experiment_A}の帳票画像Aに対して、試作したツールを適用せず、PhotolizeのGUIツールのみを用いて電子フォーム内記入欄を配置する。
次に、図\ref{fig:experiment_B}の帳票画像Bに対して、試作したツールを適用し、修正対象の電子フォーム内記入欄を修正し、電子フォーム内記入欄を配置する。
グループ$\beta$は、図\ref{fig:experiment_A}の帳票画像Aに対して、試作したツールを適用し、修正対象の電子フォーム内記入欄を修正し、電子フォーム内記入欄を配置する。
次に、図\ref{fig:experiment_B}の帳票画像Bに対して、試作したツールを適用せず、PhotolizeのGUIツールのみを用いて電子フォーム内記入欄を配置する。

被験者のグループ分けと実験手法を、表\ref{tb:experiment_case}に示す。
\begin{table}[tp]
	\centering
	\caption{被験者のグループ分けと実験手法}
    \label{tb:experiment_case}
    \begin{tabular}{cc}
        \begin{minipage}[c]{0.45\hsize}
            \centering
            \begin{tabular}{c|c|c}
                グループ & 被験者 & 学年\\
                \hline \hline
                \multirow{3}{*}{グループ$\alpha$} & 被験者1 & 修士2年生\\
                                               & 被験者2 & 修士1年生\\
                                               & 被験者3 & 学部4年生\\
                                        \hline
                \multirow{3}{*}{グループ$\beta$} & 被験者4 & 修士1年生\\
                                              & 被験者5 & 修士1年生\\
                                              & 被験者6 & 学部4年生
	        \end{tabular}
        \end{minipage} &
        \begin{minipage}[c]{0.45\hsize}
            \centering
            \begin{tabular}{c|c|c}
                グループ & 帳票画像 & 手法 \\
                \hline \hline
                \multirow{2}{*}{グループ$\alpha$} & 帳票画像A & GUIツールのみ \\
                                               & 帳票画像B & 試作ツール + 修正 \\
                                        \hline
                \multirow{2}{*}{グループ$\beta$} & 帳票画像A & 試作ツール + 修正 \\
                                              & 帳票画像B & GUIツールのみ
            \end{tabular}
        \end{minipage}
    \end{tabular}
\end{table}
なお、試作したツールで取得した領域座標、および、ラベルについては、間違っている可能性がある。
修正対象とする電子フォーム内記入欄を、以下に示す。

\begin{itemize}
    \item 誤検出によって、帳票画像内記入欄が存在しない場所に、電子フォーム内記入欄を配置したもの
    \item 帳票画像内に存在する帳票画像内記入欄を検出できず、電子フォーム内記入欄を配置できていないもの
    \item 本来想定するラベルとは異なるラベルを割り付けたもの
\end{itemize}

実験で計測する時間について、試作したツールを適用し、JSONファイルと2枚の領域強調画像を出力するまでの時間を、実行時間と呼ぶ。
また、電子フォーム内記入欄の配置が完了するまでの時間を、配置時間と呼ぶ。

実験にあたり、PhotolizeのGUIツールを用いて、電子フォーム内記入欄を配置する。
Photolizeの操作画面を、図\ref{fig:photolize}に示す。
\begin{figure}[tp]
    \begin{center}
        \includegraphics[width=15cm]{image/06-discussion/photolize.jpg}
        \caption{Photolizeの操作画面}
        \label{fig:photolize}
    \end{center}
\end{figure}
Photolizeは、メインウインドウとサイドバーで構成されている。
メインウインドウには、背景とする帳票画像が映される。
配置した電子フォーム内記入欄については青色、配置済みの電子フォーム内記入欄のうち、移動や拡縮などの操作を対象とするものは赤色で表示される。
サイドバーには、配置する電子フォームのラベルの種類を示す複数のボタンがある。
これらのボタンを、メインウインドウの帳票画像上にドラッグすることで、電子フォーム内記入欄を配置する。

PhotolizeのGUIツールを用いて、電子フォーム内記入欄を配置する手順を、図\ref{fig:photolize_how_to_use}に示す。
\begin{figure}[tp]
    \begin{center}
        \includegraphics[width=15cm]{image/06-discussion/photolize_how_to_use.jpg}
        \caption{PhotolizeのGUIツールで電子フォーム内記入欄を配置する操作手順}
        \label{fig:photolize_how_to_use}
    \end{center}
\end{figure}
例えば、「個数」を記入する帳票画像内記入欄には、「数値」の電子フォーム内記入欄を選択し、ウインドウに表示する帳票画像内にドラッグすることで、電子フォーム内記入欄を配置できる。
配置した電子フォーム内記入欄については、矩形であり、各頂点と各辺の中点(図\ref{fig:photolize_how_to_use}内の電子フォーム記入欄上の白い正方形)をドラッグすることで、拡縮を行う。
配置済みの電子フォーム記入欄のうち、各頂点と各辺の中点でない箇所をドラッグすることで、移動を行う。
なお、Photolizeを利用するにあたり、記入する内容に適する種類の電子フォーム内記入欄を選択して配置することで、ラベルを考慮したとみなす。
ラベルの種類については、試作したツールにおける日付(date)、文字列(string)、数値(number)に対応するよう、それぞれ「日付/時刻入力」、「単一行テキスト」、「数値」の3種類のみを用いる。

試作したツールを適用し、PhotolizeのGUIツールで電子フォーム内記入欄を配置した結果を修正する場合の実験手順を、以下に示す。

\begin{enumerate}
    \item 開始前に、実験者が被験者に対して、実験対象の帳票画像とPhotolizeについて説明し、Photolizeにおける電子フォーム内記入欄の配置操作も併せて説明する。
    \item 実験者が実行時間を計測を開始し、試作したツールを実行する。
    \item 実験者が、試作したツールがJSONファイルと2枚の画像ファイルを出力したことを確認し、実行時間の計測を終了する。
    \item 実験者が配置時間の計測を開始し、被験者にPhotolizeのGUIツールを用いて、試作したツールが配置した電子フォーム内記入欄を修正させる。
    \item 実験者が全ての電子フォーム内記入欄を正しく配置したことを確認し、配置時間の計測を終了する。
\end{enumerate}

試作したツールを適用せず、PhotolizeのGUIツールのみを用いて、電子フォーム内記入欄を配置する場合の実験手順を、以下に示す。

\begin{enumerate}
    \item 開始前に、実験者が被験者に対して、実験対象の帳票画像とPhotolizeについて説明し、Photolizeにおける電子フォーム内記入欄の配置操作も併せて説明する。
    \item 実験者が配置時間の計測を開始し、被験者にPhotolizeのGUIツールのみを用いて、電子フォーム内記入欄を配置させる。
    \item 実験者が全ての電子フォーム内記入欄を正しく配置したことを確認し、配置時間の計測を終了する。
\end{enumerate}

\subsection{電子フォーム内記入欄を配置するまでにかかる時間に関する評価}\label{subsec:evalue_required_time}
本節では、電子フォーム内記入欄を配置するまでにかかる時間について、評価を行う。

帳票画像A、Bについて、被験者6名の各計測時間とその合計時間を、表\ref{tb:result_time}に示す。
\begin{table}[tp]
    \caption{被験者が電子フォーム内記入欄を配置するまでにかかる時間(分:秒)}
	\label{tb:result_time}
    \centering
    \begin{tabular}{ccc||rrr} 
    グループ & 被験者 & 帳票画像 & 実行時間 & 配置時間 & 合計時間 \\
    \hline \hline
    
    \multirow{6}{*}{グループ$\alpha$} & \multirow{2}{*}{被験者1} & 帳票画像A & - & 08:49 & 08:49 \\ % 宮下さん M2  
                                                            & & 帳票画像B & 01:10 & 03:49 & 04:59 \\ 

                                    & \multirow{2}{*}{被験者2} & 帳票画像A & - & 10:03 & 10:03 \\ % 柿木さん M1
                                                            & & 帳票画像B & 01:30 & 02:50 & 04:20 \\
                                                            
                                    & \multirow{2}{*}{被験者3} & 帳票画像A & - & 07:32 & 07:32 \\ % 田中くん B4
                                                            & & 帳票画像B & 01:12 & 03:18 & 04:30 \\
                                                            

                                                            \hline
    \multirow{6}{*}{グループ$\beta$} & \multirow{2}{*}{被験者4} & 帳票画像A & 01:26 & 02:48 & 04:11 \\ % 高倉さん M1
                                                            & & 帳票画像B & - & 07:11 & 07:11 \\
                                                            
                                    & \multirow{2}{*}{被験者5} & 帳票画像A & 01:20 & 02:38 & 03:58 \\ % 翁長さん M1
                                                            & & 帳票画像B & - & 07:32 & 07:32 \\ 
                                                            
                                    & \multirow{2}{*}{被験者6} & 帳票画像A & 01:33 & 02:21 & 03:54 \\
                                                            & & 帳票画像B & - & 06:56 & 06:56 \\ 

    \end{tabular}
\end{table}
また、表\ref{tb:result_time}のうち、帳票画像Aにおける各計測時間の平均を、表\ref{tb:result_imageA_mean_time}に示す。
さらに、表\ref{tb:result_time}のうち、帳票画像Bにおける各計測時間の平均を、表\ref{tb:result_imageB_mean_time}に示す。
\begin{table}[tp]
	\centering
    \begin{tabular}{cc}
        \begin{minipage}[c]{0.5\hsize}
            \centering
            \caption{帳票画像Aにおける各計測時間の平均(分:秒)}
            \label{tb:result_imageA_mean_time}
            \begin{tabular}{c|rrr}
                グループ & 実行時間 & 配置時間 & 合計時間 \\
                \hline \hline
                グループ$\alpha$ & - & 08:48 & 08:48 \\
                グループ$\beta$ & 01:26 & 02:36 & 04:02 \\
	        \end{tabular}
        \end{minipage} &
        \begin{minipage}[c]{0.5\hsize}
            \centering
            \caption{帳票画像Bにおける各計測時間の平均(分:秒)}
            \label{tb:result_imageB_mean_time}
            \begin{tabular}{c|rrr}
                グループ & 実行時間 & 配置時間 & 合計時間 \\
                \hline \hline
                グループ$\alpha$ & 01:12 & 03:19 & 04:31 \\
                グループ$\beta$ & - & 07:13 & 07:13 \\
            \end{tabular}
        \end{minipage}
    \end{tabular}
\end{table}
合計時間を算出する式を、式\ref{eq:sum_time}に示す。
なお、PhotolizeのGUIツールのみを用いて枠を配置した場合は、実行時間は0秒として計算する。

\begin{equation}\label{eq:sum_time}
    合計時間=実行時間+配置時間
\end{equation}

表\ref{tb:result_imageA_mean_time}より、帳票画像Aに関する合計時間は、試作したツールを使用したグループ$\alpha$が、PhotolizeのGUIツールのみを用いたグループ$\beta$よりも平均で4分46秒(約54.17\%)短い。
同様に、表\ref{tb:result_imageB_mean_time}より、帳票画像Bに関する合計時間は、試作したツールを使用したグループ$\beta$が、PhotolizeのGUIツールのみを用いたグループグループ$\alpha$よりも平均で2分42秒(約37.41\%)短い。
以上の結果から、試作したツールは、電子フォーム内記入欄を配置する時間の削減について、有用であることがわかった。

% \begin{table}[t]
% 	\caption{ケース$\alpha$の被験者が配置完了までにかかる時間(分:秒)}
% 	\label{tb:result_caseA_time}
% 	\centering
% 	\begin{tabular}{cc||rrrr|r}
% 		被験者 & 帳票画像 & 実行時間 & 配置時間 & 合計時間 \\
%         \hline \hline

% 		% 宮下さん M2
% 		\multirow{2}{*}{被験者1} & 帳票画像A & 00:00 & 08:49 & 08:49 \\
% 		                        & 帳票画像B & 01:10 & 03:49 & 04:59 \\
%                                 \hline

% 		% 田中くん B4
% 		\multirow{2}{*}{被験者2} & 帳票画像A & 00:00 & 07:32 & 07:32 \\
%                                 & 帳票画像B & 01:12 & 03:18 & 04:30 \\
%                                 \hline

% 		% 柿木さん M1
% 		\multirow{2}{*}{被験者3} & 帳票画像A & 00:00 & 10:03 & 10:03 \\
%                                 & 帳票画像B & 01:30 & 2:50 & 04:20 \\
%                                 \hline \hline

% 		% 平均
% 		\multirow{2}{*}{平均}  & 帳票画像A & 00:00 & 08:48 & 08:48 \\
%                               & 帳票画像B & 01:12 & 03:19 & 04:31 \\
% 	\end{tabular}
% \end{table}

% \begin{table}[t]
% 	\caption{ケース$\beta$の被験者が配置完了までにかかる時間(分:秒)}
% 	\label{tb:result_caseB_time}
% 	\centering
% 	\begin{tabular}{rc||rrrr|r}
% 		被験者 & 帳票画像 & 実行時間 & 配置時間 & 合計時間 \\
%         \hline \hline

% 		% 高倉さん M1
% 		\multirow{2}{*}{被験者4} & 帳票画像A & 01:26 & 02:48 & 04:11 \\
%                                 & 帳票画像B & 00:00 & 07:11 & 07:11 \\
%                                 \hline

% 		% 翁長さん M1
% 		\multirow{2}{*}{被験者5} & 帳票画像A & 01:20 & 02:38 & 03:58 \\
%                                 & 帳票画像B & 00:00 & 07:32 & 07:32 \\
%                                 \hline

% 		%
% 		\multirow{2}{*}{被験者6} & 帳票画像A & &  &  \\
%                                 & 帳票画像B & 00:00 &  &  \\
%                                 \hline \hline

% 		% 平均
% 		\multirow{2}{*}{平均}   & 帳票画像A &  &  &  \\
%                                & 帳票画像B & 00:00 &  &  \\
% 	\end{tabular}
% \end{table}



\subsection{配置した電子フォーム内記入欄の精度に関する評価}\label{subsec:evalue_accuracy}
本節では、配置した電子フォーム内記入欄の精度について評価する。
なお、精度の評価は、領域座標とラベルについての、それぞれの適合率と再現率を算出することで行う。

それぞれの適合率、および、再現率を算出する式を、以下に示す。
なお、正しくラベルを割り付けた電子フォーム内記入欄は、配置した電子フォーム内記入欄の位置が正しいことを前提とする。

\begin{equation}\label{eq:area_precision}
    領域座標の適合率=\frac{正しく配置した電子フォーム内記入欄の数}{出力した領域座標の数}
\end{equation}

\begin{equation}\label{eq:area_recall}
    領域座標の再現率=\frac{正しく配置した電子フォーム内記入欄の数}{帳票画像内記入欄の数}
\end{equation}

\begin{equation}\label{eq:label_precision}
    ラベルの適合率=\frac{正しくラベルを割り付けた電子フォーム内記入欄の数}{出力した領域座標の数}
\end{equation}

\begin{equation}\label{eq:label_recall}
    ラベルの再現率=\frac{正しくラベルを割り付けた電子フォーム内記入欄の数}{帳票画像内記入欄の数}
\end{equation}

領域座標の適合率と再現率の平均を、表\ref{tb:result_rect_accuracy}に示す。
また、ラベルの適合率と再現率の平均を、以下の表\ref{tb:result_label_accuracy}に示す。
\begin{table}[tp]
    \centering
    \begin{minipage}[h]{0.47\linewidth}
        \caption{出力した領域座標の適合率と再現率}
        \label{tb:result_rect_accuracy}
        \centering
        \begin{tabular}{r|c|c}
            帳票画像 & 適合率 & 再現率 \\
            \hline \hline
            帳票画像A & 0.806 & 1.000 \\
            帳票画像B & 0.857 & 0.875 \\
        \end{tabular}
    \end{minipage}
    \begin{minipage}[h]{0.47\linewidth}
        \caption{出力したラベルの適合率と再現率}
        \label{tb:result_label_accuracy}
        \centering
        \begin{tabular}{r|c|c}
            帳票画像 & 適合率 & 再現率 \\
            \hline \hline
            帳票画像A & 0.731 & 0.907 \\
            帳票画像B & 0.674 & 0.688 \\
        \end{tabular}
    \end{minipage}
\end{table}
表\ref{tb:result_rect_accuracy}、および表\ref{tb:result_label_accuracy}より、2枚の帳票画像に共通して、領域座標およびラベルの再現率は、適合率に比べ高いことがわかる。
式\ref{eq:area_precision}、および式\ref{eq:label_precision}より、適合率は、不要に領域座標を取得する回数が少ないほど上昇し、不要に領域座標を取得する回数が多いほど低下する。
また、式\ref{eq:area_recall}、および式\ref{eq:label_recall}より、再現率は、正しく配置した電子フォームの数が多いほど上昇し、正しく配置した電子フォームの数が少ないほど低下する。
よって、再現率は適合率に比べ高いことから、試作したツールが配置した電子フォーム内記入欄は、配置する必要がある場所については正しく配置できる一方で、誤検出が多く、不要な電子フォーム内記入欄を配置していることがわかる。
この結果より、試作したツールを用いて修正する場合は、電子フォーム内記入欄を配置する修正の回数よりも、不要に配置した電子フォーム内記入欄を削除する修正の回数が多いと推測できる。
電子フォーム内記入欄を配置する場合は、ラベルを考慮しつつ、位置を調整する必要がある。
一方、電子フォーム内記入欄を削除する場合は、その必要がないため、電子フォーム内記入欄を配置する場合と比較して、手間がかからない。
よって、試作したツールは、電子フォーム内記入欄を配置する手間の削減について、有用であることがわかった。

しかし、表\ref{tb:result_rect_accuracy}、および表\ref{tb:result_label_accuracy}より、領域座標とラベルについて、領域座標の適合率を除き、帳票画像Bにおける適合率と再現率は、帳票画像Aにおける適合率と再現率と比較して低い。
これについては、帳票画像Bに、矩形および下線部で示されていない記入欄を含むことと、帳票画像Bを撮影した環境によって、二値化の結果に影響があり、一部の領域を検知できなかったことが原因である。
これらの問題については、\ref{sec:problems}節で後述する。

% \begin{equation}
%     ラベルの適合率=\frac{正しくラベルを割り付けた電子フォーム内記入欄の数}{出力した領域座標の数}
% \end{equation}

% \begin{equation}
%     ラベルの再現率=\frac{正しくラベルを割り付けた電子フォーム内記入欄の数}{帳票画像内記入欄の数}
% \end{equation}



% 領域座標について、精度に関する評価指標をまとめた表を、表\label{tb:result_accuracy}に示す。

% \begin{table}[t]
% 	\caption{本ツールが出力した領域座標の精度に関する評価指標}
% 	\label{tb:result_accuracy}
% 	\centering
% 	\begin{tabular}{cc||rrrr|r}
% 		帳票画像 & 配置する電子フォーム内記入欄の数 & 適合率 & 再現率 & 正解率 \\
%         \hline \hline
%         帳票画像A & 54 & 00:00 & 08:49 & 08:49 \\
%         帳票画像B & 48 & 01:10 & 03:49 & 04:59 \\
%         \hline
% 	\end{tabular}
% \end{table}

% \begin{table}[t]
% 	\caption{ケース$\alpha$の被験者が正しく配置した電子フォーム内記入欄}
% 	\label{tb:result_accuracy}
% 	\centering
% 	\begin{tabular}{cc||rrrr|r}
% 	     & 再現率 & 適合率 & 正解率 & 合計時間 \\
%         \hline \hline

% 		% 宮下さん M2
% 		\multirow{2}{*}{被験者A} & 帳票画像A & 00:00 & 08:49 & 08:49 \\
% 		                        & 帳票画像B & 01:10 & 03:49 & 04:59 \\
%                                 \hline

% 		% 
% 		\multirow{2}{*}{被験者B} & 帳票画像A & 00:00 &  &  \\
%                                 & 帳票画像B &  &  &  \\
%                                 \hline

% 		%
% 		\multirow{2}{*}{被験者C} & 帳票画像A & 00:00 & 00:19 &  \\
%                                 & 帳票画像B & 00:57 &  &  \\
%                                 \hline

% 		%
% 		\multirow{2}{*}{被験者D} & 帳票画像A & 00:00 & 00:19 &  \\
%                                 & 帳票画像B & 00:57 &  &  \\
%                                 \hline \hline

% 		% 平均
% 		\multirow{2}{*}{平均}  & 帳票画像A & 00:00 &  &  \\
%                               & 帳票画像B &  &  &  \\
% 	\end{tabular}
% \end{table}




\section{関連ツール}\label{sec:relation_tools}
本節では、本研究で試作した記入欄自動検出およびラベル割付ツールと、関連ツールを比較する。
電子フォーム作成ツールに、株式会社シムトップスのi-Reporter\cite{i-Reporter}や、インフォテック株式会社のCreate!Form\cite{Create!Form}がある。
これらの既存の電子フォーム作成ツールは、帳票をExcelファイルとして管理している場合は、電子フォームを簡単かつExcelファイルのレイアウトとセルの書式を保って作成できる。
一方、帳票を紙媒体で管理している場合は、既存の電子フォーム作成ツールを使用するために、帳票をExcelファイルとして作成し、入力とするか、紙媒体の帳票を撮影した画像ファイルを入力として、電子フォームを手作業で作成する必要があり、それぞれ以下の2つの課題がある。

\begin{itemize}
    \item Excelファイルを入力とする場合、紙媒体のレイアウトをもとにExcelファイルを作成する必要があり、使い慣れた既存の帳票のレイアウトが変わる場合がある。
    \item 画像ファイルを入力とする場合、画像を背景として、GUIツールを用いたマウス操作で記入欄を配置する必要があるため、電子フォームの作成に手間と時間がかかる。
  \end{itemize}

また、サイボウズ社が提供するクラウドサービスであるkintone\cite{kintone}を利用し、データベースに保存したデータから、記入済みの帳票を自動作成する外部連携サービスに、トヨクモ株式会社のPrintCreator\cite{PrintCreator}や、オーサムジョブ合同会社のk-Report\cite{k-Report}がある。
しかし、背景となるPDFファイルを用意し、電子フォーム内記入欄をGUIツールのマウス操作で配置することによって、帳票のテンプレートを作成する必要があり、手間と時間がかかる。
このように、既存の電子フォーム作成ツールやサービスは、これらの課題を同時に解決できない。

試作したツールは、レイアウトを変更せず、帳票画像内記入欄を自動検出し、バリデーションチェックに必要である記入内容のデータ型をラベルとして割り付け、JSONファイルとして出力できる。
また、JSONファイルの内容を人間が確認しやすくするため、自動検出した帳票画像内記入欄と、そのラベルを強調表示して、画像として出力できる。
これにより、試作したツールを適用することによって、レイアウトを変更せず、電子フォーム作成の時間と手間を削減できる。

\section{試作したツールの問題点}\label{sec:problems}
試作したツールの問題点を、以下に示す。

\begin{itemize}
    \item 記入内容を示す欄を、帳票画像内記入欄として検出する\\
        矩形領域と下線部領域は、それぞれ矩形と直線を検出し、領域座標を取得したものである。
        領域座標取得部(\ref{sec:area_coords_obtainment_part}節を参照)において、文字情報を参照しないため、領域座標が記入する内容を示す欄か、帳票画像内記入欄かを判別することができない。
        特に、矩形が隣接する形式の帳票である際、行または列が記入する内容を示す場合がある。
        これによって、記入する内容を示す欄の数だけ、不要に領域座標を出力してしまう。
        この問題点については、取得した領域座標と文字位置を参照し、領域座標の中心点から一定の範囲内であれば、出力から除外することで解決できると考える。
    \item 割り付けるラベルが不安定である\\
        試作したツールにおいて、割り付けるラベルは、文字情報とYouriの出力に依存する。
        Tesseract-OCRは、光学文字認識の結果が必ずしも正しいとは限らず、本来認識するべき文字とは別の文字として認識する場合がある。
        例えば、「は」と「ば」などの視覚的な違いが少ない文字や、「品」と「口」などの一部に別の漢字を含む漢字が存在する場合は、誤認識する可能性が高い。
        この誤認識によって、\ref{sec:result_rect}節で述べたような、誤ったラベルを割り付ける場合がある。
        また、Youriの出力は、誤って属性を推測する場合があり、かつ出力が常に一定ではない。
        Tesseract-OCRとYouriの2つの不安定な出力によって決定するため、割り付けるラベルが不安定である。
        この問題点については、他の光学文字認識ソフトウェアの利用や、プロンプトの改善によって解決できると考える。
    \item 特殊なレイアウトの帳票画像においては、精度が低下する\\
        欄の一部に色が付いているもの、絵や押印を含むものなど、記入欄と文字以外のものが帳票画像にある場合は、領域座標を取得する精度と、ラベルを割り付ける精度の両方が低下する。
        これは、帳票画像内記入欄、二値化の結果に影響を及ぼすためである。
        特に、青や紫などの暗色は、グレースケール画像において画素値が高くなり、二値化した際に黒となる可能性が高い。
        二値化によって、記入欄と文字以外に黒の画素が存在する場合、誤検出の可能性が高くなる。
        この問題点については、二値化手法の変更や、一定範囲内に閾値を超える数の黒の画素を認識した場合は、隣接する黒色の画素を白色に変換することで解決できると考える。
    \item 記入内容を示す文字が帳票画像内記入欄の右にある場合は、他の文字の属性をラベルとして割り付けてしまう\\
        \ref{subsec:label_link_processing}節で述べたラベルを割り付ける手法は、記入内容を示す文字が帳票画像内記入欄の左にある場合のみに正常に動作する。
        記入内容を示す文字が帳票画像内記入欄の右にある場合もラベルの更新を行った場合、行ごとに共通したラベルを割り付けてしまうため、1列ごとに記入内容を決定する形式の帳票に対しては、正常にラベルを割り付けることができない。
        この問題点については、帳票の記入方向を検知し、検知した結果によってラベルを更新する順番を変更することで解決できると考える。
    \item 記入欄の一部が、矩形または下線部で表示されていない場合は、領域座標を取得できない\\
        帳票画像内記入欄は、矩形または下線部で示されていることを前提とするため、文字のみで帳票画像内記入欄を示すものについては、領域座標を取得できない。
        この問題点については、光学文字認識の結果、品詞が名詞である単語のみを認識した場合は、その単語のバウンディングボックスと同じ矩形を右に配置することで解決できると考える。
    \item 撮影環境によって、矩形と下線部の検出、および文字認識の精度が低下する\\
        図\ref{fig:experiment_B}のような電子化文書の画像を入力とした場合は、電子文書の画像と比較して、矩形と下線部の検出、および文字認識の精度が低い傾向にある。
        これは、DeblurGANv2で除去できなかったブレや、撮影場所の明るさ、撮影時に映る影などの撮影環境が二値化の結果に影響を及ぼすためである。
        この問題点については、陰影除去などの別の画像処理を加えることで解決できると考える。
\end{itemize}
\chapter{おわりに}\label{cha:Conclusion}
本研究では、電子フォーム作成時間の短縮を目的として、帳票画像内における記入欄の領域座標、および、その記入欄に入力する内容のデータ型をまとめたJSONファイルに加えて、検知した記入欄を強調表示した画像2枚を出力する、記入欄の検出およびラベル付与手法を提案した。
本提案手法では、帳票画像を入力として、領域座標とラベルをまとめたJSONファイルと、JSONファイルの内容に対応し、領域とラベルを帳票画像に描画した2枚の領域強調画像を出力とするツールの作成を手段とした。

適用例を用いて、本ツールが領域座標の取得と、ラベル割付が一部を除いて正常に動作することを確認した。
なお、正常に動作しない原因は、文字認識が失敗したことによって、想定とは異なるラベルを領域座標に割り付けたことや、記入する内容を示す文字が、帳票画像記入欄の右にあることによって、領域座標の右にある文字の属性を参照しないためであることを確認した。
また、JSONファイルの内容と、2枚の領域強調画像が正しく対応した状態で出力することを確認した。

本ツールの有用性を評価するため、実験を行い、電子フォーム記入欄を配置するまでにかかる時間を削減できたことを確認した。
また、本ツールの出力について、精度を確認するため、領域座標とラベルの、それぞれの適合率と再現率を評価し、削減した時間に対して、修正する必要がある電子フォーム記入欄の割合が妥当であることを確認した。

以上のことから、本研究で提案した手法は、電子フォーム作成にかかる手間と時間の削減に有用であると言える。

今後の課題を、以下に示す。

\begin{itemize}
    \item 記入欄と記入内容を示す欄の区別\\
        特に矩形が隣接する形式の帳票である場合、行や列の先頭に、記入内容を示す矩形の欄があるときがある。
        これにより、帳票画像記入欄ではない矩形の領域座標を不要に取得してしまうため、対応する必要がある。
    \item 領域座標に割り付けるラベルの安定\\
        領域座標に割り付けるラベルは、文字認識と大規模言語モデルの出力に依存する。
        現在のTesseract-OCRでは、文字を100\%正確に認識することができず、大規模言語モデルの出力は実行の度に変化する。
        これにより、電子フォーム記入欄に想定とは異なるラベルを割り付けてしまう場合があるため、対応する必要がある。
    \item レイアウトによる精度の低下\\
        背景や帳票画像記入欄の一部に色が付いている、押印があるなどの二値画像は、黒の画素の数が増加する。
        これにより、帳票画像記入欄の検出やラベル割付の精度が低下してしまう場合があるため、対応する必要がある。
    \item 記入内容を示す文字の判定\\
        ある帳票画像記入欄について、記入する内容を示す文字が右にある場合、ラベルを更新することができない。
        これにより、想定とは異なるラベルを割り付けてしまう場合があるため、対応する必要がある。
\end{itemize}

% \section{はじめに}\label{cha:Introduction}
% \tool 作ったよ。
% 先行研究\cite{example}を参考にしたよ。

% \section{\tool の機能}

% \section{\tool の適用例}
% \tool にソースコード\ref{lst:input-example}を入力した結果を、図\ref{image/sample.png}に示す。
% \begin{figure}[hb]
% \begin{lstlisting}[label={lst:input-example}, caption={入力例}]
% coordinate00,0,0
% coordinate01,1,1
% coordinate02,2,2
% coordinate03,3,3
% \end{lstlisting}
% \end{figure}

% \begin{figure}[hb]
%     \centering
%     \includegraphics[scale=0.5]{image/sample.png}
%     \caption{\tool の画面}
%     \label{image/sample.png}
% \end{figure}

% \section{\tool の評価}


% \section{まとめ}

% \footnotesize

%%
% 謝辞
%
\acknowledgment
本研究の遂行と、本論文の執筆にあたり、終始熱心なご指導ご鞭撻を賜りました、宮崎大学工学部情報システム工学科の片山徹郎教授に、心より感謝申し上げます。

また、本研究にあたり、多大なお力添えを賜りました、codeless technology株式会社の皆様に、深く感謝申し上げます。

最後に、片山徹郎研究室の皆様に、感謝いたします。

特に教授と先輩方は、研究方針の相談を聞いてくださり、論文の添削も遅い時間までしていただきました。ありがとうございました。

%%
%参考文献
%
\begin{thebibliography}{0}
  \bibitem{example}Example Org: "Example Title". \url{http://example.com}. アクセス日: 2021/01/31.
\end{thebibliography}

% \newpage
% \listoftodos
\end{document}
